\documentclass{article}
\usepackage{graphicx} % Required for inserting images
\graphicspath{{figures/}}
\usepackage{array}
\usepackage{caption}
\usepackage{hyperref}
\usepackage{setspace}
\usepackage{bookmark}
\usepackage{geometry}
\usepackage{listings}
\usepackage{subcaption}
\usepackage[T1]{fontenc}
\usepackage[parfill]{parskip}
%\usepackage{multirow}

% provvisorio
\title{AGENT-BASED MODELING AND LEARNING FOR EPIDEMIOLOGICAL STUDIES}
\author{m.stievano1@campus.unimib.it}

\pagenumbering{roman}
\begin{document}
% sistemare spaziature e quindi impaginazione
\begin{titlepage}

    \noindent
    \begin{minipage}[t]{0.19\textwidth}
        \vspace{-4mm}{\includegraphics[scale=1.15]{img/logo_unimib.pdf}}
    \end{minipage}
    \begin{minipage}[t]{0.81\textwidth}
    {
            \setstretch{1.42}
            {\textsc{Università degli Studi di Milano - Bicocca}} \\
            \textbf{Scuola di Scienze} \\
            \textbf{Dipartimento di Informatica, Sistemistica e Comunicazione} \\
            \textbf{Corso di Laurea Magistrale in Informatica} \\
            \par
    }
    \end{minipage}

\vspace{20mm}

\begin{center}
        {\LARGE{
                \setstretch{1.2}
                % Titolo tesi provvisorio
                \textbf{AGENT-BASED MODELING AND LEARNING FOR EPIDEMIOLOGICAL STUDIES}
                \par
        }}
    \end{center}

    \vspace{60mm}

    \noindent
    {\large \textbf{Relatore:} Prof. Antoniotti Marco} \\

    \noindent
    {\large \textbf{Correlatore:} Prof. }

    \vspace{10mm}

    \begin{flushright}
        {\large \textbf{Relazione della prova finale di:}} \\
        \large{Matteo Stievano} \\
        \large{Matricola 829535}
    \end{flushright}

    \vspace{10mm}
    \today
    \begin{center}
        {\large{\bf Anno Accademico 2022-2023}}
    \end{center}

    \restoregeometry

\end{titlepage}

\begin{abstract}
    % inserire abstract della tesi. 
    % forse deve essere sia in eng che ita
    Lo scopo di questa tesi è quello di studiare il comportamento di modelli di simulazione
    e di intervento utilizzando il linguaggio di programmazione Julia, e i suoi framework
    Agents.jl, SciML.ai. In particolare viene proposto l'utilizzo del framework Agents.jl come 
    metodo per la simulazione di sistemi complessi e il framework SciML.ai come base per 
    lo sviluppo di un controllore tramite tecniche ibride di machine learning, come ad esempio 
    l'utilizzo delle Neural ODE. Lo scopo principale è quello di avere un approccio ibrido e dinamico 
    ai due mondi, dai quali è possibile trarre il massimo vantaggio. L'approccio generale 
    è stato modellato come un problema complesso che lavora con una struttura dati a grafo, simulando
    una rete sociale, in cui i nodi del grafo caratterizzano gli agenti del modello. Per ottenere un 
    miglioramento delle prestazioni e dell'affidabilità, il modello è stato ibridato con 
    un sistema di ODE. Il controllore si basa sull'idea di una Neural ODE che controlla 
    in maniera autonoma e automatica il livello di contromisure applicate. Queste vengono 
    identificate come una riduzione dell'indice di infettività $R_0$; per evitare di applicare 
    livelli di contromisure al lato pratico insostenibili, è stato inserito un valore di contenimento 
    chiamato \emph{happiness}, il quale agisce come una funzione di costo e controllo delle contomisure stesse.
\end{abstract}

% se mai volessi scrivere una piccola dedica
\include{./txt/dedica}

% indice dei contenuti
\tableofcontents
\newpage
\listoffigures
\newpage
% \listoftables
% \newpage

\pagenumbering{arabic}
% introduzione al testo
\section{Introduzione}

L'impiego di metodologie e tecniche sempre più avanzate è stato oggetto 
di discussione e interesse nella comunità scientifica, in particolare tra 
epidemiologi e medici. Negli ultimi anni, il mondo è diventato notevolmente
 interconnesso, aumentando significativamente la probabilità di diffusione 
 globale di virus e di conseguenti catastrofi senza precedenti.

La storia umana è segnata da epidemie, ma solo alcune di esse hanno 
lasciato un'impronta duratura nella memoria collettiva a causa delle loro 
conseguenze catastrofiche. Tra queste, alcune delle più note includono la 
peste nera che nel XIV secolo mieté venti milioni di vite in Europa in 
soli sei anni, l'epidemia di tifo durante le crociate e la Seconda 
Guerra Mondiale, l'influenza spagnola che causò 50 milioni di morti tra 
il 1918 e il 1920, e l'epidemia di AIDS, che ha colpito oltre 75 milioni 
di persone e causato 35 milioni di decessi dal 1981.

Oggi, l'influenza stagionale non suscita più lo stesso timore nei 
cittadini dei paesi sviluppati, ma la pandemia di COVID-19, iniziata alla 
fine del 2019, ha condizionato l'umanità per tre anni e continua a farlo, 
causando finora quasi 7 milioni di vittime accertate. Questa pandemia 
rimarrà impressa nella memoria umana poiché ha messo in crisi l'intero 
sistema governativo globale, causando allarmi, panico e, talvolta, 
isteria, come pochi altri eventi sono stati in grado di fare.

Esaminando le statistiche di questa epidemia, i numeri relativi ai decessi 
e agli infetti (quasi 7 milioni di morti e oltre 700 milioni di infetti) 
da soli sono sufficienti a preoccupare qualsiasi lettore. Tuttavia, ci 
sono altri dati meno evidenti che forniscono informazioni altrettanto 
inquietanti, come l'impatto economico globale e il concetto di 
"poverty trap", un circolo vizioso in cui la povertà e le malattie 
perpetuano un ciclo di bassa salute e crescente povertà.

Questi sono solo alcuni dei problemi che possono emergere durante una 
pandemia globale come quella del COVID-19, e quindi la comunità 
scientifica, in particolare gli epidemiologi, cerca costantemente 
soluzioni più efficaci ed accurate per prevenire, contenere e gestire 
eventi di questo genere.

L'epidemiologia è una disciplina relativamente giovane che si è evoluta 
per affrontare emergenze sanitarie. La sua definizione originale risale 
al 1978, ma nel corso del tempo è cresciuta e si è adattata alle esigenze 
della società. Attualmente, l'epidemiologia è definita come lo studio 
della distribuzione e dei determinanti delle condizioni o eventi legati 
alla salute in una specifica popolazione, con l'applicazione di questo 
studio al controllo dei problemi di salute.

Uno strumento ampiamente utilizzato in epidemiologia è la simulazione 
tramite software, che si basa su modelli matematici per trarre conclusioni 
sui sistemi analizzati. Questi sistemi, spesso definiti complessi, 
coinvolgono molteplici componenti che interagiscono tra loro e possono 
essere descritti mediante modelli matematici.

I primi modelli epidemiologici erano compartimentali, basati su gruppi di 
individui separati che interagivano tra loro e potevano essere descritti 
mediante equazioni differenziali ordinarie (ODE). Uno dei modelli più noti 
è il modello Suscettibile-Infetto-Ricoverato (SIR) sviluppato da 
Kermack e McKendrick nel 1927. Successivamente, sono emersi i modelli ad 
agenti, che sono autonomi e consentono la simulazione di sistemi complessi 
tramite l'interazione di entità autonome.

La simulazione ha fatto notevoli progressi dagli anni '90 grazie 
all'espansione delle risorse computazionali. Tuttavia, i modelli di 
simulazione richiedono compromessi e semplificazioni, ad esempio la 
discretizzazione degli ambienti di simulazione. Inoltre, è necessario 
considerare il comportamento umano in situazioni di pericolo, il che può 
essere complesso da modellare.

L'identificazione delle cause di un fenomeno e la comprensione di come un 
intervento influenzi tale fenomeno rappresentano una delle sfide più 
complesse dell'epidemiologia. La causalità, ossia la relazione di 
causa-effetto tra eventi o variabili, non è sempre evidente e richiede 
un'analisi rigorosa. La correlazione tra eventi non implica 
necessariamente causalità.

In questa introduzione, abbiamo fornito una panoramica sull'epidemiologia, 
la simulazione di sistemi complessi e la sfida della causalità. 
Le sezioni successive analizzeranno lo stato dell'arte dell'epidemiologia 
e della simulazione, con un focus sulla pandemia da COVID-19 e sui metodi 
di monitoraggio e intervento nelle simulazioni epidemiologiche.

% descrizione dell'attuale stato delle cose
% epidemiologia
\section{Panoramica dello Stato dell'Arte}

\subsection{Epidemiologia}
L'epidemiologia rappresenta una disciplina biomedica di fondamentale 
importanza, focalizzata sull'analisi della distribuzione e dell'incidenza 
delle malattie e degli eventi sanitari rilevanti all'interno di una 
popolazione. Questo campo di studio si dedica all'approfondimento 
delle cause, dei pattern temporali e delle conseguenze delle malattie 
\cite{Galea2009-lj} \cite{Parascandola2001-kw}.

L’epidemiologia si divide in quattro ambiti principali \cite{rothman2015modern}: epidemiologia
descrittiva, epidemiologia analitica, epidemiologia clinica ed epidemiologia
sperimentale.

\begin{enumerate}
    \item \textbf{L’epidemiologia descrittiva} si occupa di studiare la
    frequenza e la distribuzione delle malattie e dei parametri di salute
    nelle popolazioni. Descrive eventi sanitari come malattie, cause di morte 
    la presenza di fattori di rischio come, ad esempio, il fumo di tabacco,
    l’inquinamento atmosferico.
    \item \textbf{L’epidemiologia analitica} si occupa invece di studiare le
    cause delle malattie e degli eventi sanitari. In particolare, cerca di 
    identificare i fattori che possono influenzare l’insorgenza o lo sviluppo
    della malattia.
    \item \textbf{L’epidemiologia clinica} si concentra sullo studio dei
    pazienti affetti da una determinata patologia. In particolare, cerca di
    identificare i fattori che possono influenzare il decorso della malattia
    e l’efficacia dei trattamenti.
    \item \textbf{L’epidemiologia sperimentale} si occupa invece di studiare 
    gli effetti delle terapie preventive o curative sulla popolazione.
\end{enumerate}

L'epidemiologia si caratterizza per la sua natura essenzialmente 
pratica, poiché il suo obiettivo principale consiste nel determinare 
le cause sottese a un determinato effetto sanitario. Questa disciplina 
si trova ad affrontare una serie di sfide complesse che definiscono 
il suo percorso. Tra queste sfide possiamo trovare l'identificazione 
delle relazioni di causalità tra eventi.

Tali interrogativi possono sembrare elementari, dato che, come specie, 
abbiamo sviluppato un istinto innato nell'individuare correlazioni 
causali tra eventi, anche quando queste non esistono effettivamente. 
Ad esempio, se ci trovassimo in un bosco buio, soli e circondati 
solo dal sussurro di una leggera brezza estiva e dovessimo udire un 
rumore provenire dai cespugli, è probabile che lo assoceremmo a un 
pericolo imminente, come la presenza di un predatore, sebbene il 
rumore sia causato dalla brezza stessa.

Questo adattamento evolutivo ci ha permesso di sopravvivere in 
situazioni di pericolo, ma purtroppo, quando si tratta di scienza e dati, 
l'istinto non è sempre una guida affidabile. I dati, per loro natura, 
sono neutri e non trasmettono automaticamente informazioni significative; 
spetta a noi, in quanto individui dotati di intelligenza, 
tecniche e metodi, estrarre significato da questi dati in modo accurato 
ed inequivocabile.

Un'osservazione chiave è che ciò che sembra ovvio può essere fuorviante. 
Per esempio, il grafico seguente sembra dimostrare in modo 
"inequivocabile" una relazione diretta tra la spesa degli Stati Uniti 
per la ricerca aerospaziale e il numero di suicidi per strangolamento:

\begin{figure}[H]
    \begin{center}
        \includegraphics[width=\linewidth]{img/chart.png}
        \caption{Esempio di correlazione spuria}
        \url{https://www.tylervigen.com/spurious-correlations}
        \label{fig:spurious_relations}
    \end{center}
\end{figure}

Sulla base di questo grafico e dei dati presentati, si potrebbe 
erroneamente concludere che le due categorie sono in qualche modo 
correlate e che il governo degli Stati Uniti debba essere 
accusato di incitare al suicidio. Tuttavia, questo è un esempio 
di una relazione spuria, in cui due o più variabili sono 
associate ma non sono causalmente collegate.

È importante sottolineare che l'associazione e la causalità non sono 
la stessa cosa, e quando si studia una, è fondamentale non confonderla 
con l'altra. In statistica, una correlazione tra dati rappresenta 
qualsiasi tipo di relazione tra due o più variabili, indipendentemente 
dalla sua natura causale o non causale. Nel caso precedente, 
la correlazione tra le due variabili potrebbe essere semplicemente 
dovuta al passare del tempo: nel corso degli anni, la spesa media per 
la ricerca aerospaziale è continuata a crescere a causa di un interesse 
e di investimenti sempre maggiori in quel settore, mentre nel tempo 
si è verificato un costante aumento del numero di suicidi.

Il nostro pregiudizio verso la ricerca di collegamenti tra eventi, 
in modo che sembrino sempre collegati in modo tangibile e che si 
possa tracciare una chiara e distinta linea causale dall'inizio alla 
fine, può ingannarci quando tali collegamenti sembrano evidenti ma 
non lo sono. Spesso, la spiegazione più semplice è anche la meno 
interessante, sebbene sia corretta: due eventi completamente svincolati 
tra loro possono avere andamenti simili, e differenti fattori possono 
portare allo stesso comportamento. 

Il problema della causalità è di estrema importanza e rappresenta 
una delle sfide principali quando si cercano di sviluppare e 
applicare interventi all'interno di una popolazione per mitigare, 
ad esempio, la diffusione di un agente patogeno \cite{Parascandola2001-kw}.

Come già accennato, i dati per se stessi sono privi di significato; 
è il nostro compito imparare a interpretare il loro significato. 
Un esempio eloquente di come, nonostante la consapevolezza del problema 
delle correlazioni spurie, i dati possano comunque trarre in inganno 
è il seguente:

Supponiamo di essere medici e di dover decidere se prescrivere un 
certo farmaco a un paziente. Per prendere questa decisione, abbiamo 
a disposizione la storia clinica del paziente e i risultati di uno 
studio su un nuovo farmaco che sembra promettente nel trattamento 
della sua malattia. Questo farmaco è stato testato su un gruppo di 
700 persone, suddivise in due sottogruppi di 350 pazienti ciascuno. 
I pazienti hanno scelto autonomamente se assumere o meno il farmaco.
Ecco i risultati:

\begin{table}[H]
    \centering
    \caption{Paradosso di Simpson}
    \begin{tabular}{ |p{2.2cm}||p{1.6cm}|p{1.6cm}|p{1.6cm}||p{1.6cm}|p{1.6cm}|p{1.6cm}| }
        \hline
        \multicolumn{7}{|c|}{Paradosso di Simpson} \\
        \hline
        Categoria & Pazienti & Guariti & \% Guariti & Pazienti & Guariti & \% Guariti\\
        \hline
        Uomini & 87 & 81 & 93\% & 270 & 234 & 87\% \\
        Donne & 263 & 192 & 73\% & 80 & 55 & 69\% \\
        Dati combinati & 350 & 273 & 78\% & 350 & 289 & 83\% \\
        \hline
    \end{tabular}
\end{table}

Questi risultati sembrano suggerire che la prescrizione di questo 
nuovo farmaco non abbia un impatto positivo sulla guarigione dei 
pazienti. Tuttavia, questo risultato è in realtà un esempio del 
cosiddetto "paradosso di Simpson", in cui i dati aggregati relativi 
a un trattamento specifico sembrano indicare una perdita di efficacia, 
mentre i dati delle singole categorie mostrano risultati opposti. 
Questo esempio sottolinea il fatto che l'interpretazione di dati 
aggregati non sempre può essere affidabile, e talvolta può ingannare. 
In questi casi, è necessario estrarre le informazioni sulla causalità 
dai dati individuali.

È evidente che comprendere le cause di un determinato effetto o 
insieme di effetti non è un compito banale. Anche conoscendo 
l'agente patogeno o almeno la sua natura, non sempre è sufficiente 
per spiegare la complessità delle interazioni. 
L'utilizzo di modelli di apprendimento automatico per l'analisi dei 
dati, la ricerca di correlazioni e la successiva formulazione di 
politiche di intervento può rappresentare un rischio, ma al contempo 
offre un alleato potente nella definizione di politiche di intervento 
in settori estremamente delicati come quello della sanità 
\cite{doi:10.1098/rsos.220638}.
% modelli compartimentali: deterministici e stocastici
\subsection{Modelli Compartimentali}
% scrivere stato dell'arte simile a chapter 2 pg.33 pdf book (download)
In epidemiologia i modelli compartimentali sono una tecnica di modellezione 
generica che si predispone molto bene allo studio complessivo del comportamento
di una malattia infettiva \cite{wiki:Compartmental_models_in_epidemiology}. 
Questa tecnica di modellazione si applica anche ad altre branche della 
scienza, come ad esempio la finanza.

Questa tecnica di modellazione matematica basa il proprio funzionamento 
sull'assunzione che, data una popolazione di individui, questi vengano 
etichettati in maniera differente, in base allo stato di progressione 
della malattia che hanno, o non hanno, contratto. Così facendo si vanno a 
definire dei compartimenti ben separati che possono interagire tra loro, ma 
che rimangono comunque chiaramente distinti l'uni dagli altri.

Il modello che tutt'ora viene usato come riferimento e come base per 
lo studio e modellazione è il così detto modello 
\textbf{Susceptible, Infectious, Recovered} (SIR):

\begin{figure}[h]
    \includegraphics[width=\linewidth]{img/sir.png}
    \caption{Struttura modello SIR} 
    \label{fig:SIR_Structure}
\end{figure}

Questo modello è stato ideato all'inizio del 20esimo secolo, 
più precisamente nel 1917, da Kermack e McKendrick. Come introdotto questo modello
si basa sull'assunzione che all'interno di una popolazione durante 
il decorso di una malattia vi possano esistere solamente tre stadi in cui 
un individuo può essere inserito: 

\begin{itemize}
    \item Susceptible: Questo stadio rappresenta lo stato iniziale per la maggior parte
    degli individui all'interno di una popolazione. Rappresenta il numero di 
    persone che possono contrarre la malattia.
    \item Infectious: Questo stadio rappresenta tutti quegli individui che dallo 
    stato di Susceptible, dopo essere venuti in contatto con un individui infetto, 
    diventano a loro volta individui infetti.
    \item Recovered: Questo stadio rappresenta una duplice categoria, quella degli
    individui che alla fine del docorso della malattia sopravvivono ad essa, e 
    quelli che invece muoiono a causa di questa. Generalemente questo stato viene
    anche definito come Removed.
\end{itemize}

Da questa semplice idea poi si è andato a sviluppare un modello
matematico per descrivere come queste 3 categorie separate ma 
che si influenzano vicendevolmente, cambiano nel corso del tempo.
Questo approccio si basa sull'utilizzo di un sistema di Equazioni 
Ordinarie Differenziali (ODE) \cite{Brauer2008}. Una ODE è
un equazione differenziale, ovvero un equazione che lega 
una funzione incognita alle sue derivate, che coinvolge una 
funzione di una variabile e le sue derivate di ordine qualsiasi.
Questo oggetto viene utilizzato estensivamente in molti ambiti 
della scienza e in epidemiologia viene utilizzato per 
descrivere un sistema dinamico \cite{wiki:Equazione_differenziale_ordinaria}. 

\begin{figure}[h]
    \includegraphics[width=\linewidth]{img/SIR-model.png}
    \caption{Visualizzazione grafica di un modello SIR} 
    \label{fig:SIR_model_graphic}
\end{figure}

Come succede per la maggior parte di tutte le equazioni 
differenziali, queste non possono essere solitamente 
risolte in maniera esatta, e per questo ci si limita a 
studiarne il comportamento qualitativo della soluzione 
senza essere capaci di ottenere una'espressione analitica.

Nell'ambito epidemiologico tuttavia non sempre l'utilizzo
di un sistema di ODE è preferito come metodo generale di 
modellazione di un sistema dinamico, in quanto il più 
delle volte questo sistema al suo interno utilizza un 
insieme di variabili rappresentanti processi stocastici, le quali 
è bene mantenere tali. A questo scopo vengono utilizzati i 
sistemi di Equazioni Differenziali Stocastiche (SDE) \cite{Allen2008}.

Queste equazioni si basano sulla teoria del moto Browniano 
il quale descrive il movimento randomico delle particelle 
sospese all'interno di un medium \cite{wiki:Brownian_motion}. 
In questo modo è possibile modellare in maniera più granulare
ad esempio la diffusione di un agente patogeno tramite il 
medium aereo, come può essere il COVID-19.

\subsubsection{Derivazioni del modello SIR}

Con il tempo questo il modello SIR è stato espanso 
per tenere in considerazione comportamenti
differenti sia della popolazione che delle malattie, 
andando a definire una moltitudine di modelli utili a 
differenti scopi. In epidemiologia il modello di
riferimento maggiormente utilizzato è il modello SEIR 
(Susceptible, Exposed, Infectious, Recovered)
con le sue varianti proprie di ogni approccio.

Questa tipologia di modello basa il proprio punto di forza
sull'osservazione che da quando un individuo viene 
infettato tramite un agente patogeno a quando quest'ultimo 
diventa infettivo, passa un periodo di latenza in cui 
l'individuo non può ne infettare ne essere infettato. 
Questo periodo viene anche conosciuto come \emph{periodo di 
incubazione} \cite{wiki:Incubation_period}. Con questa 
conoscenza pregressa è possibile sviluppare modelli 
e policy che tenendo conto di questo comportamento lo 
sfruttino per arginare l'epidemia.

\begin{figure}[h]
    \includegraphics[width=\linewidth]{img/SEIR-compartmental-model-schematic.png}
    \caption{Modello schematico SEIR} 
    \label{fig:SEIR_model}
\end{figure}

Alcuni modelli definiscono i propri stati in maniera da considerare 
come stato interno al sistema anche l'agente patogeno, così da 
poter modellare e simulare l'andamento dell'infettività della pandemia 
in relazione alle contromisure prese, siano esse farmaceutiche o non.
Ne è un esempio il modello proposto da \cite{Mwalili2020} nel quale 
il modello viene proposto con l'idea di incorporare le misure di 
distanziamento sociale come variabili per misurare la loro efficacia
contro la recente pandemia da COVID-19.

\begin{minipage}{\linewidth}
    \centering
    \includegraphics[width=\textwidth]{img/13104_2020_5192_Fig1_HTML.png}
    \captionof{figure}{Esempio di modello SEIR preso dall'articolo \cite{Mwalili2020}}
    \label{fig:SEIR_model_social_distancing}
\end{minipage}

Altri modelli, come quello proposto da \cite{ijerph17103535} mantengono 
la stessa filosofia, ovvero quella di analizzare l'efficacia delle misure 
di prevenzione non farmaceutiche sull'andamento di un epidemia, ma non
modellano esplicitamente l'agente patogeno come stato del modello, bensì
variando i paramentri di infettività e contagio, arrivano allo stesso 
risultato. Un'altra differenza tra i due approcci è quella della tipologia
di equazioni differenziali utilizzate, \cite{Mwalili2020} hanno utilizzato 
delle ODE mentre \cite{ijerph17103535} delle SDE.

\begin{minipage}{\linewidth}
    \centering
    \includegraphics[width=\textwidth]{img/ijerph-17-03535-g001.png}
    \captionof{figure}{Esempio di modello SEIR preso dall'articolo \cite{ijerph17103535}}
    \label{fig:SEIR_model_article}
\end{minipage}

Il motivo per cui viene utilizzato il modello SEIR come base è perchè permette di 
modellare una caratteristica intrinseca di una malattia infettiva, ovvero 
il periodo di latenza che un individuo appena infettato ha prima di diventare 
infettivo a sua volta e mostrare i sintomi di infezione. Questo permette 
di osservare quanto le misure di sicurezza e prevenzione non farmaceutiche 
sono efficaci sulla popolazione tenendo in considerazione 
un tempo di ritardo intrinseco nel feedback tra l'attuamento delle 
misure di prevensione e i risultati positivi di queste ultime.

Una delle modifiche più utilizzate a questo modello è quella di avere un sistema 
ciclico, ovvero in cui gli individui che entrano nello stato R non diventano immuni 
alla malattia a tempo indefinito, ma perdono questa loro caratteristica di immunità
dopo un dato periodo di tempo. Questo permette di modellare con più accuratezza le malattie
infettive stagionali come ad esempio la comune influenza o il raffreddore, oppure 
mostrare l'andamento ad ondate di altre malattie che hanno la caratteristica di 
mutare molto velocemente, come è stato per il COVID-19 e le sue innumerevoli varianti.

\begin{minipage}{\linewidth}
    \centering
    \includegraphics[width=\textwidth]{img/41592_2020_856_Fig1_HTML.png}
    \captionof{figure}{Modello SEIRS preso dall'articolo \cite{Bjornstad2020}}
    \label{fig:SEIRS_model}
\end{minipage}

Questa variante denominata SEIRS permette invece di osservare, non solo l'andamento e 
l'efficacia delle contromisure non farmaceutiche, ma anche di quelle
farmaceutiche, come ad esempio i vaccini; o più in generale l'andamento
della così detta immunità di gregge \cite{Bjornstad2020}. 

Rimanendo sull'idea di voler analizzare l'efficacia di un vaccino, una
modifica comune al modello SEIR è quella legata all'aggiunta dello stato V,
Vaccinated, come stato esplicito oppure oppure implicito al modello. Questa 
variazione permette di modellallare con più attenzione l'efficacia di un 
vaccino una volta introdotto all'interno della popolazione, 
ma più in generale permette di osservare l'efficacia di una politica di vaccinazione 
in relazione al numero di vaccinazioni effettuate in un determinato periodo di tempo.
Questo viene solitamente affiancato con un modello ciclico, così da poter
osservare come bisogna modificare le proprie politiche vaccinali in vista
di ondate cicliche più o meno intense di infezioni.

\begin{figure}[h]
    \begin{center}
        \includegraphics[scale=0.6]{img/nuovi_positivi_2023-04-21.pdf}
        \caption{Esempio di ondate di infettività. Dati del Dipartimento di Protezione Civile Italiana}
        \label{fig:DPC_new_positive}
    \end{center}
\end{figure}

\begin{minipage}{\linewidth}
	\centering
	\includegraphics[width=\textwidth]{img/cumulative_plot_2023-06-11.pdf}
	\captionof{figure}{Grafico cumulativo dei dati della pandemia da COVID-19 in Italia. Dati ottenuti da Our World in Data}
	\label{fig:voc}
\end{minipage}

Ai fini pratici di una simulazione avere uno stato esplicitamente definito
oppure ricavabile dalle probabilità di transizione degli altri stati 
è pressocchè indifferente, e potrebbe essere richiesta una diferenziazione
solamente in caso in cui si avrebbe una differenza sostanziale tra lo stato
R e V, ad esempio in termini di protezione dalla malattia, durata immunità etc... .

% \begin{figure}[h]
%     \begin{center}
%         \includegraphics[width=\linewidth]{img/seirv_explicit.jpg}
%         \caption{Esempio di modello SEIRV con stato esplicito per la condizione V}
%         \label{fig:SEIRV_explicit}
%     \end{center}
% \end{figure}

\begin{figure}[h]
    \begin{center}
        \includegraphics[width=\linewidth]{img/seirv_implicit.jpg}
        \caption{Esempio di modello SEIRV con stato implicito per la condizione V}
        \label{fig:SEIRV_implicito}
    \end{center}
\end{figure}

Non essendoci un numero massimo di stati, e quindi di equazioni, utilizzabili
all'interno del modello, ogni individuo è libero di definire un numero
di equazioni arbitrario che rispecchia la sua idea di modellazione del 
sistema. Ne è un esempio il modello riportato in \cite{Giordano2020}.

Come precedentemente introdotto esistono due grandi famiglie di modelli
per la simulazione, e sono rispettivamente la famiglia di modelli deterministici
e quella di modelli stocastici.

\subsubsection{Modelli Deterministici}
I modelli deterministici vengono principalmente utilizzati 
per la loro immediatezza e riproducibilità. Infatti un modello
deterministico, una volta impostati i parametri necessari
riprodurrà sempre lo stesso risultato. Questo tipo di approccio, 
seppur utilizzatto in larga scala come ad esempio da \cite{Bjornstad2020}
\cite{Mwalili2020}, \cite{Giordano2020} si basa su delle assunzioni molto forti
che non sempre rispecchiano la realtà. 

Infatti i modelli deterministici hanno il grosso problema di
essere affidabili solamente nel caso in cui vi siano dati 
sufficientemente grandi, cosa che non sempre è possibile
avere \cite{wiki:Compartmental_models_in_epidemiology}.
Essendo modelli di tipo deterministico, avendo dei parametri 
di infettività maggiori di zero, con un numero di individui infetti
anch'esso maggiore di zero, si tenderà ad avere nel lungo
periodo un andamento di equilibrio endemico derivato 
dalle equazioni e dal modello utilizzato. Questo comportamento
però non sempre rispetta la realtà, ma come precedentemente 
accennato, in casi in cui si hanno grandi quantità di dati 
legati principalmente alla popolazione, questi modelli si 
comportano in maniera affidabile.

\begin{minipage}{\linewidth}
    \centering
    \includegraphics[width=\textwidth]{img/3-s2.0-B9780128012383988378-f98837-02-9780128012383.jpg}
    \captionof{figure}{Esempio equilibrio endemico}
    \label{fig:Endemic_equilibrium}
\end{minipage}

\subsubsection{Modelli Stocastici}
I modelli stocastici, seppur più complessi e non determinabili 
a priori, permettono una modellazione più veritiera e 
simile alla realtà in quanto tengono in considerazione 
variazioni randomiche che possono capitare durante il 
decorso di una pandemia. Tuttavia questa loro caoticità 
richiede che per ottenere risultati robusti debbano essere 
eseguiti e computati molteplici volte, e la media dei loro 
risultati è il valore vero da tenere in considerazione. 
Questi modelli sono stati applicati durante la pandemia da COVID-19 
come ad esempio da \cite{ijerph17103535}. 

\begin{minipage}{\linewidth}
    \centering
    \includegraphics[width=\textwidth]{img/Gillespie-e1643395123662.png}
    \captionof{figure}{Modello SIR stocastico}
    \label{fig:Endemic_equilibrium_stochastic_sir}
\end{minipage}

E' immediato notare come il comportamento delle curve sia 
imprevedibile se preso singolarmente, e che non sempre 
esiste uno stato di equilibrio endemico chiaro e definito come
quello ottenibile da un modello deterministico. Ciò nonostante 
effettuando molte simulazioni è possibile vedere come il 
comportamento generale del modello sia comunque simile a 
quello di un modello deterministico.
% modelli ad agente: discreti e continui in spazio e tempo
\subsection{Modelli ad Agente}
Descrizione di cosa sia una simulazione e perchè è importante effettuarla. Descrizione della intrinseca randomicità di quest'ultima e dei vantaggi che essa può
portare. Affrontare i vari problemi che una simulazione può avere ma anche i 
vantaggi che essa può offrire. 
Descrivere come alcuni approcci uniscano il concetto di modellazione matematica
dei modelli compartimentali ai modelli ad agente creando degli ibridi (es abm + PDE)

\subsubsection{Discretizzazione}
Definire cosa vuole dire discretizzare qualcosa. Differenti approcci alla 
discretizzazione con tanto di pro e contro. 
Introduzione alla discretizzazione del tempo e dello spazio e dei modelli 
che ne possono beneficiare (gridspace, graphspace, continuousspace, etc...)

% julia
\subsection{Julia}

%\begin{figure}[h]
%    \includegraphics[width=\linewidth]{img/Julia_Programming_Language_Logo.svg.png}
%    \caption{Logo del linguaggio di programmazione Julia}
%    \label{fig:Julia_logo}
%\end{figure}

Julia è un linguaggio di programmazione ad alto livello, 
multi-paradigma e open-source ideato per compiere analisi 
numerica ed effettuare operazioni di computer science in 
maniera rapida e stabile. Julia è nato ufficialmente come 
linguaggio di programmazione nell’anno 2012 con lo scopo di 
fornire uno strumento potente, robusto e veloce tanto se non 
più dei linguaggi considerati in questo ambito lo stato 
dell’arte, ovvero C e Fortran;  ma anche facile da approcciare, 
al contrario dei linguaggi sopra citati. Julia è un linguaggio 
di programmazione scritto in C++ e Scheme, ma gran parte della 
sua composizione è scritta in Julia stesso 
\cite{wiki:Julia_(programming_language)}.

le caratteristiche principali di questo linguaggio sono 
principalmente:
\begin{itemize}
    \item Alte performance: lo scopo per cui Julia è nato è 
    stato quello di offrire un linguaggio estremamente 
    performante con la capacità di poter compilare programmi 
    in codice nativo per molteplici piattaforme grazie 
    all’utilizzo di LLVM

    % \begin{figure}[h]
    %     \begin{center}
    %         \includegraphics[width=\linewidth]{img/julia_llvm.jpg}
    %         \caption{Esempio struttura Julia e LLVM}
    %         \label{fig:Julia_LLVM}
    %     \end{center}
    % \end{figure}

    \item Dinamico: la scelta di rendere Julia un linguaggio 
    dinamicamente tipizzato lo rende di facile utilizzo in 
    quanto rende molto più semplice il suo approccio anche a 
    chi non ha una base solida di programmazione, in quanto 
    ritorna la stessa sensazione di immediatezza di un 
    linguaggio di scripting. Inoltre questo permette un alto 
    supporto per l’uso interattivo

    % \begin{figure}[h]
    %     \includegraphics[width=\linewidth]{img/typing_example.jpg}
    %     \caption{Esempio differenze di typing in alcuni linguaggi di programamzione}
    %     \label{fig:Different_typing}
    % \end{figure}

    \item Ambiente riproducibile: lo scopo del linguaggio è 
    quello di poter permettere all’utente di ricreare le 
    stesse condizioni ogni volta su ogni macchina su cui un 
    programma viene eseguito. Questo può essere ottenuto 
    tramite l’utilizzo di file binari pre compilati
    \item Componibile: Julia utilizza l’approccio multiple 
    dispatch come paradigma, permettendo una grande 
    flessibilità nell’esprimere una elevata quantità di 
    pattern di programmazione, dall’object-oriented al 
    funzionale
    \item General Purpouse: lo scopo del linguaggio è quello 
    di creare un ecosistema in grado di poter soddisfare 
    qualsiasi esigenza di un utente, permettendo la creazione 
    di applicativi e microservizi senza dover ricorrere ad 
    integrazioni con codice non nativo Julia
    \item Open source: Julia abbraccia la filosofia open source, 
    e il codice sorgente dell’intero linguaggio, così come di 
    tutte le librerie è disponibile sulla piattaforma GitHub 
    sotto la licenza MIT. Questo permette una crescita 
    eterogenea grazie al contributo di più di 1000 utenti 
    che si impegnano a migliorare il linguaggio
\end{itemize}

\subsubsection{Agents.jl}

%\begin{minipage}{\linewidth}
%    \centering
%    \includegraphics[width=\textwidth]{img/Agents_5poOwRo.png}
%    \captionof{figure}{Logo framework Agents.jl}
%    \label{fig:Agents.jl_logo}
%\end{minipage}

Seguendo la filosofia propria del linguaggio di programmazione 
in cui è sviluppata, la libreria Agents.jl \cite{Agents.jl} 
viene sviluppata con l’obiettivo di essere facile da imparare e 
usare ed estendibile, con forte attenzione sulla creazione ed 
evoluzione di modelli veloci e soprattutto scalabili. 
Molteplici esempi comparativi sono stati effettuati mostrando 
come il framework sviluppato permetta di avere un notevole 
guadagno prestazionale rispetto ai maggiori competitor 
attualmente presenti sul mercato (Mesa, Netlogo, MASON) 
\cite{ABAR201713}.

\begin{minipage}{\linewidth}
    \centering
    \includegraphics[width=\textwidth]{img/1-s2.0-S1574013716301198-gr1_lrg.jpg}
    \captionof{figure}{Tabella comparativa}
    \label{fig:Comparative_table}
\end{minipage}

La facilità di interazione con questa libreria non è da 
confondersi con una mancanza di opzioni durante lo sviluppo, 
in quanto nativamente Agents.jl permette l’integrazione con 
altre librerie che in maniera altrettanto semplice e veloce 
offrono all’utente la possibilità 
di addentrarsi nel mondo del machine learning, in particolar 
modo il mondo del Scientific Machine Learning 
\cite{rackauckas2017differentialequations}, 
branca che soprattutto grazie alla pandemia da Covid-19 ha 
visto un sempre piu' crescente interesse. 

Agents.jl offre molteplici opzione di configurazione, ma 
principalmente quello su cui si basa sono i seguenti principi:

\begin{itemize}
    \item definizione di un tipo di agente, generalmente viene raccomandato di 
    estendere la tipologia \emph{StandardABM} la quale e' la piu' concreta implementazione,
    nonche' l'implementazione di default, di un costruttore generico di un \textbf{AgentBasedModel}.
    \item definizione di una tipologia di spazio, esistono principalmente 2 tipologie 
    di spazio da poter utilizzare come base e si basano sull'utilizzo di uno spazio \emph{discreto}
    oppure \emph{continuo}.
    \begin{itemize}
        \item spazio discreto a grafo: un \emph{GraphSpace} rappresenta uno spazio del modello
        rappresentato da un grafo arbitrario in cui ogni nodo puo' contenere una 
        quantita' di agenti abitraria. Per funzionare correttamente questa tipologia di 
        spazio richiede che gli agenti implementino al loro interno specifici attributi
        per rappresentare la loro posizione all'interno dello spazio. Questa tipologia di spazio
        si appoggia alla libreria \textbf{Graphs.jl} \cite{Graphs2021} per gestire tutte le operazioni relative
        alla struttura dati del grafo.  

        \begin{minipage}{\linewidth}
            \centering
            \includegraphics[width=\textwidth]{img/graph.png}
            \captionof{figure}{Rappresentazione di uno spazio a grafo}
            \label{fig:graphspace_representation}
        \end{minipage}
        
        \item spazio discreto a griglia: un \emph{GridSpace} rappresenta uno spazio del modello
        rappresentato da una griglia di dimensione D $\geq$ 1. Questa tipologia di spazio 
        richiede l'utilizzo di una metrica per la definizione della distanza
        tra celle di una griglia. Ci sono attualmente tre tipologie di metriche supportate 
        e sono: \emph{Euclidean}, \emph{Manhattan} e \emph{Chebyshev}.

        \begin{minipage}{\linewidth}
            \centering
            \includegraphics[width=\textwidth]{img/distance.png}
            \captionof{figure}{Metriche di distanza di una griglia}
            \label{fig:gridspace_distances}
        \end{minipage}
        
        \item spazio continuo: un \emph{ContinuousSpace} rappresenta uno spazio di dimensione
        D $\in$ (0, $\infty$). E' fortemente consigliato di attribuire ad un agente all'interno 
        di questo spazio due caratteristiche fondamentali, una posizione e una velocita'. Questa 
        tipologia di spazio permette di rappresentare delle proprieta' spaziali tramite valori 
        finiti oppure tramite \emph{funzioni}, i quali rappresentano una discretizzazione di 
        valori spaziali che potrebbero non essere disponibili in maniera analitica. Utilizzando questa 
        tipologia di spazio la metrica di distanza utilizzata sara' sempre \emph{Euclidian}.
        Per velocizzare il calcolo della posizione degli agenti, viene effettuata una discretizzazione
        implicita dello spazio, ma questa puo' essere forzata a rimanere nello spazio continuo 
        ottenendo un calo di prestazioni.

        \item spazio misto: un \emph{OpenStreetMapSpace} rappresenta una mappa come un'entita' 
        continua che preferisce l'accuratezza alle prestazioni. La mappa viene rappresentata 
        come un grafo connesso. I nodi non rappresentano necessariamente intersezioni. 
    \end{itemize}
\end{itemize} 


\subsubsection{SciML.ai}

\begin{minipage}{\linewidth}
    \centering
    \includegraphics[width=\textwidth]{img/SciMLGitHubPreview.png}
    \captionof{figure}{Logo SciML.ai}
    \label{fig:SciML.ai}
\end{minipage}

SciML.ai è una collezione di librerie dedite all'analisi numerica e 
e al calcolo scentifico. Questo framework permette di 
avere tutti gli strumenti per poter utilizzare facilmente, 
velocemente e in maniera robusta tecniche di analisi numerica 
molto avanzata, così da poter sviluppare applicazioni complesse 
in maniera semplice e concreta
\cite{rackauckas2017differentialequations} 
\cite{rackauckas2019diffeqflux} 
\cite{rackauckas2020universal}. 

Il principale utilizzo che e' stato fatto di queste librerie si 
concentra principalmente sull'implementazione di metodi di analisi 
numerica, come ad esempio l'utilizzo di di risolutori per sistemi 
di \emph{Equazioni Ordinarie Differenziali} (ODE) che si possono 
trovare nel package \emph{OrdinaryDiffEq.jl} \cite{rackauckas2017differentialequations} 
uniti a metodi di \emph{Machine Learning} (ML) \cite{pal2023lux} \cite{Flux.jl-2018} \cite{innes:2018}
per lo sviluppo di un modello di \emph{Scientific Machine Learning}
\cite{rackauckas2019diffeqflux} \cite{rackauckas2020universal}. 
La principale differenza che esiste tra i modelli classici di ML e quelli 
di SciML e' che i secondi richiedono un numero molto meno elevato di 
dati per la comprensione delle dinamiche che li governano, rendendo questi 
modelli moltio piu' scalabili.

\begin{minipage}{\linewidth}
    \centering
    \includegraphics[width=\textwidth]{img/uode_cont.png}
    \captionof{figure}{Esempio di Scientific Machine Learning}
    \label{fig:SciML_example}
\end{minipage}

\subsubsection*{Equazioni Ordinarie Differenziali}
Nell'ambito matematico, una \emph{equazione ordinaria differenziale} (ODE) e' un 
equazione differenziale (DE) dipendente da un singolo valore indipendente, 
generalmente il tempo. All'interno di questa grande famiglia di equazioni, 
il gruppo delle \emph{equazioni lineari differenziali} gioca un ruolo predominante
in quanto la maggior parte dei fenomeni fisici e di matematica applicata possono 
essere descritti dalla soluzione di questo tipo di equazioni. 

Una equazione lineare differenziale e' definita da un \emph{polinomio lineare} 
e la sua derivata e' un equazione dalla forma:

$$\alpha_0(x)y + \alpha_1(x)y' + \alpha_2(x)y'' + ... + \alpha_n(x)y^{(n)} + b(x) = 0$$

dove $\alpha_0(x), ..., \alpha_n(x)$ e $b(x)$ sono funzioni differenziabili arbitrarie che non 
richiedono di essere lineari, e $y', ..., y^{(n)}$ sono le successive derivate della funzione incognita
$y$ della variabile $x$.

L'utilizzo di \emph{equazioni non lineari differenziali} puo' essere 
generalmente approssimato con la controparte lineare cosi' da ottenere una 
soluzione piu' semplice. 

La suite di SciML.ai offre molteplici framework per la risoluzione di sistemi di 
equazioni lineari differenziali prevalentemente all'interno delle librerie 
\emph{DifferentialEquations.jl} \cite{rackauckas2017differentialequations}
\cite{rackauckas2019confederated} \cite{9622796} \cite{gowda2019sparsity},
questi sono separati nelle seguenti categorie:

\begin{itemize}
    \item \emph{Standard Non-Stiff ODEs Solver}
    \item \emph{Standard Stiff ODEs Solver}
    \item \emph{Second Order and Dynamical ODEs Solver}
    \item \emph{IMEX Solvers}
    \item \emph{Semilinear ODEs Solver}
    \item \emph{DAE Solver}
    \item \emph{Misc Solver}
\end{itemize}

\subsubsection*{Equazioni Differenziali Universali}
Un \emph{equazione differenziale differenziale} (UDE) e' una \emph{equazione algebrica
differenziale}\cite{wiki:Differential-algebraic_system_of_equations} 
non triviale, ovvero un sistema di equazioni che contiene delle equazioni differenziali
ed equazioni algebriche oppure e' un sistema equivalente,
con la proprieta' che la sua soluzione puo' approssimare
\emph{qualsiasi} funzione continua su un qualunque intervallo $\in R$ a 
qualsiasi livello di precisione desiderata. \cite{wiki:Universal_differential_equation}

Per essere precisi, una equazione differenziale (possibilmente in forma implicita)
$P( y', y'', y''', ..., y^{(n)})=0$ e' una UDE se, per ogni funzione a valori relative
continua $f$ e per ogni funzione continua positiva $\epsilon$ esiste una 
soluzione liscia\cite{wiki:Smoothness} (una funzione e' considerabile liscia se e' 
differenziabile in ogni suo punto, percio' continua) $y$ di $P( y', y'', y''', ..., y^{(n)})=0$
con $|y(x) - f(x)| < \epsilon(x) \forall x \in R$.

Il concetto di UDE puo' essere analogo all'idea di una \emph{Macchina di Turing Universale}
\cite{wiki:Universal_Turing_machine} con la differenza che le UDE non dettano 
l'evoluzione di un sistema, ma si limitano a imporre determinate regole che 
ogni sistema che si evolve deve sodddisfare. Questo permette di avere un modello robusto 
per l'analisi di dati e la predizione dell'interazione che hanno vari fenomeni tra loro.

\begin{minipage}{\linewidth}
    \centering
    \includegraphics[scale=0.7]{img/ude_approx.png}
    \captionof{figure}{Comportamento UDE nell'approssimazione di fenomeni non lineari}
    \label{fig:UDE_approx}
\end{minipage}

Questo approccio viene spesso unitoa tecniche \emph{Data-Driven} \cite{datadrivendiffeq} per l'identificazione
sparsa di dinamiche non lineari. In particolare uno degli approcci utilizzati 
e' quello tramite l'algoritmo \emph{SINDy} \cite{wiki:Sparse_identification_of_non-linear_dynamics}. 
Questo algoritmo performa una serie di operazioni di regressione come 
ad esempio \emph{LASSO} su una libreria di funzioni candidate non lineari ottenute
da uno snapshot del sistema dinamico che si sta analizzando e delle sue derivate, 
con l'obiettivo di trovare le equazioni che lo governano. Questo procedimento 
si basa sull'assunzione che molti sistemi fisici hanno solamente una manciata di 
termini che ne dettano le dinamiche e l'evoluzione. Questo metodo e' stato largamente
utilizzato nell'identificazione della \emph{dinamica dei fluidi} cosi' come
nelle \emph{reti biologiche} e altri sistemi dinamici complessi.

\subsubsection*{Equazioni Neurali Differenziali}
Recentemente si e' iniziato a ibridare due paradigmi di modellazione come
le ODE e le reti neurali (NN) che hanno sempre avuto ambiti applicativi 
ben distinti, per cercare di ottenere il massimo da entrambe minimizzando 
gli effetti indesiderati. \cite{Kim_2021} \cite{chen2019neural}

L'idea e' quella di inserire una NN come parte sinistra di una ODE, 
e successivamente inserire a sua volta la ODE all'interno di una NN 
piu' grande. Consideriamo il seguente esempio:

$$z(0) = z_0, \frac{dz}{dt}(t) = f_\theta(t,z(t))$$

dove $z_0$ e' un qualsiasi tipo di input, mentre $f_\theta$ e' la nostra rete
neurale, e l'output del modello puo' essere successivamente utilizzato come 
input di $z(T)$ per qualche $T > 0$. Potrebbe sembrare che questo approccio sia
solamente una chimera di due tecniche distinte, quando in verita' non e' cosi. 

Utilizzando i dati ottenuti da una rete neurale, i quali sono stati estrapolati 
dai dati osservati, questi possono essere utilizzati come parametri delle equazioni
differenziali parametriche che costituiscono il modello matematico, unendo la 
flessibilita' di una rete neurale con la robustezza di un modello matematico 
di equazioni differenziali. Questo approccio vede utilizzo in molteplici applicazioni
dal \emph{deep learning} alla tradizionale modellazione matematica. 

Questo tipo di approccio e' \emph{memory efficient}, ha la capacita' di gestire
\emph{dati irregolari} con forti priori sullo spazio del modello, ha un elevata
capacita' di approssimare funzioni lineari e non lineari e si poggia su solide basi 
teoriche che pesca da entrambi i lati.

\begin{minipage}{\linewidth}
    \centering
    \includegraphics[width=\textwidth]{img/maths-oxford-ml-nde_0.png}
    \captionof{figure}{Esempio di Equazioni Differenziali Neurali}
    \label{fig:NDE_example}
\end{minipage}

\subsubsection{SafeBlues}
Durante la pandemia da Covid-19 il framework SciML.ai è stato 
utilizzato per sviluppare applicazioni le quali grazie 
all’utilizzo di tecniche di scientific machine learning 
riuscivano a prevedere in maniera molto accurata l’andamento 
dell’epidemia, seppur in presenza di una scarsa quantità di dati,
e le stesse presentavano misure di contenimento e prevenzione 
che si sono dimostrate essere efficaci nel loro utilizzo
\cite{10.1371/journal.pdig.0000142} \cite{DANDEKAR2021100220}. 

Un esempio può essere il modello denominato \textbf{SafeBlues} 
\cite{10.1371/journal.pdig.0000142} \cite{DANDEKAR2021100220} 
il quale simulando una rete bluetooth in cui gli individui potevano 
venire infettati da un virus e poi infettare a loro volta gli 
individui circostanti nella rete con una certa probabilità, 
aveva riprodotto fedelmente l’andamento della pandemia da Covid-19. 
In aggiunta questa soluzione, aveva mostrato come 
l’applicazione di policy per il contenimento del virus 
bluetooth erano perfettamente applicabili anche al caso 
reale della pandemia.

\begin{minipage}{\linewidth}
    \centering
    \includegraphics[width=\textwidth]{img/gr2.jpg}
    \captionof{figure}{Esempio funzionamento SafeBlues}
    \label{fig:SafeBlues_1}
\end{minipage}

Questa applicazione e' stata sviluppata e rilasciata per dispositivi
mobili con lo scopo di sperimentare se il sistema svilupatto (\emph{Safe Blues System})
potesse aiutare e migliorare i tradizionali metodi di previsione 
del decorso di una pandemia.

\subsubsection{Ipopt}
Ipopt (Interior Point OPTimizer) \cite{Wächter2006} e' un pacchetto software per 
l'ottimizzazione non lineare su larga scala. Questo pacchetto software e' realizzato 
per trovare delle soluzioni (locali) a problemi di ottimizzazione matematica nella forma:
$\min_{x \in R^n} f(x)$ tale che $g_L \leq g(x) \leq g_U$ e $x_L \leq x \leq x_U$, dove 
$f(x): R^n \rightarrow R$ e' la funzione obiettivo e $g(x): R^n \rightarrow R^m$ 
sono le funzioni di vincolo.

I vettori $g_L$ e $g_U$ denotano i limiti inferiore e superiore sui vincoli e i vettori
$x_L$ e $x_U$ sono i limiti delle variabili $x$. Le funzioni $f(x)$ e $g(x)$ possono essere 
sia non lineari che non convesse, ma la loro derivata seconda deve esistere e deve essere 
continua.

\begin{minipage}{\linewidth}
    \centering
    \includegraphics[width=\textwidth]{img/Comparison-of-Ipopt-performance-over-various-linear-solvers-using-the-two-dimensional.png}
    \captionof{figure}{Comparison of Ipopt performance over various linear solvers using the two-dimensional partial differential equation test problem set. \cite{unknown}}
    \label{fig:Ipopt_solver}
\end{minipage}

Ipopt e' scritto in \textbf{C++} ed e' stato rilasciato come software open source sotto la licenza \textbf{Eclipse Public License (EPL)}.


% descrizione dei metodi, strumenti e approcci utilizzati
\section{Metodi e Modelli}

\subsection{Modello ad Agente}

Descrizione approfondita del modello ad agente usato.
Spiegazione della tipologia di spazio, del perche' e' usato, e 
dell'agente. 

Spiegazione dei parametri utilizzati e del perche' sono stati 
scelti in quel modo. Spiegazione di come vado a ottenere tali parametri 
e i dati su cui faccio affidamento.

Risultati grafici.
\subsection{Monitoraggio e Intervento}

Descrizione del controllore e di come viene implementato all'interno
della logica del modello. Chiarire perche' vengono utilizzate
determinate tecniche e non altre. 

Risultati grafici.

Bug e problematiche

% discussione dei risultati ottenuti
\section{Risultati Ottenuti}

\subsection{Nessun intervento}
Il seguente grafico \ref{fig:abm_no_intervent} mostra l'andamento delle curve del modello
quando questo viene eseguito senza alcuna tipologia di intervento. Questo andamento e' mostrato 
in maniera cumulativa rispetto all'andamento dei singoli agenti, i quali possono mostrare comportamenti 
differenti tra loro.

\begin{minipage}{\linewidth}
	\centering
	\includegraphics[width=\textwidth]{img/SocialNetworkABM_NO_CONTROL.pdf}
	\captionof{figure}{Grafico cumulativo del modello senza senza alcun tipo di intervento}
	\label{fig:abm_no_intervent}
\end{minipage}

Complessivamente l'andamento del modello e' similare all'andamento standard di un modello di tipo  
SEIR, con qualche variazione dipendente dai fattori di stocasticita' intrisechi del modello; che in questo
caso non sono troppo presenti. Il grafico mostra le traiettorie piu' comuni delle curve cumulate del modello, 
dove vengono messe in evidenza i percorsi piu' utilizzati. 

Come e' possibile notare, il numero di individui suscettibili crolla drasticamente
per via della diffusione rapida e simil esponenziale che ha il virus. Questa viene emulata
dall'altrettanto rapida crescita di individui guariti (recovered) che pero', per via
di come e' stato definita la condizione di guariti, non sono immuni alle varianti del virus, permettendo 
di modellare una possibile ciclicita' dell'epidemia data dalla perdita di immunita' della popolazione. 
Queste proprieta' contribuiscono ad un andamento ciclico delle curve. Si puo' notare come 
la curva associata all'andamento degli individui nella classe \emph{D} abbia una crescita lineare, 
seppur non troppo evidente.

A seguire si puo' osservare come la curva associata alla variabile di happiness del modello,
valore che serve per bilanciare la durezza delle misure di controllo per evitare 
di cadere in un ciclo funzionale ma insostenibile, mostra un comportamento alquanto bizzarro.
Questo e' dovuto principalmente a come viene definita la funzione di controllo della felicita' \ref{fig:happiness}.
Si osserva inoltre che la curva tende ad un \emph{plateau} passato il periodo della "prima ondata".

\begin{minipage}{\linewidth}
	\centering
	\includegraphics[width=\textwidth]{img/happiness.png}
	\captionof{figure}{Definizione funzione di happiness}
	\label{fig:happiness}
\end{minipage}

Questo comportamento principalmente irrealistico e associato alla definizione che e' stato fatto della 
funzione \textbf{happiness} \ref{fig:happiness}. Il comportamento di questa curva non e' completamente 
realistico, e' comunque utilizzabile per lo scopo di mantenere sotto controllo le contromisure $\eta$.

\begin{minipage}{\linewidth}
	\centering
	\includegraphics[width=\textwidth]{img/coronavirus-data-explorer.png}
	\captionof{figure}{Grafico delle mutazioni del virus SARS-COV2 preso da Our World in Data}
	\label{fig:covid_mutation}
\end{minipage}

Infine si nota come, seppur la definizione della funzione associata alla creazione di una
nuova \emph{Variant of Concern (VOC)} \ref{fig:voc} sia semplicistica e irrealistica, 
il grafico mostra come su un periodo di circa 3 anni, associabile alla durata del periodo covid, 
il numero di VOC sia pressoche' sovrapponibile con quanto osservato dai dati raccolti durante la pandemia \ref{fig:covid_mutation}. 

\subsection{Intervento non farmaceutico}
Il seguente grafico \ref{fig:abm_intervent} mostra l'andamento delle curve del modello
quando questo viene eseguito tramite l'applicazione di una qualche tipologia di intervento non farmaceutico. 
Le contromisure sono rappresentate come un insieme di valori appartenenti all'intervallo $[0, 1)$ 
le quali rappresentano cumulativamente un insieme di metriche differenti non esplicite che vanno a definire 
un insieme di misure di controllo. Le misure di controllo cercano di mimicare quelle associate al progetto 
\textbf{OxCGRT} il quale ha lo scopo di misurare la durezza delle contomisure applicate in un determinato paese; 
questo valore e' composito di 9 metriche: chiusura delle scuole; chiusura dei luoghi di lavoro; 
cancellazione di eventi pubblici; restrizioni agli assembramenti pubblici; 
chiusura dei trasporti pubblici; obbligo di rimanere a casa; campagne di informazione pubblica; 
restrizioni agli spostamenti interni e controlli sui viaggi internazionali.

\begin{minipage}{\linewidth}
	\centering
	\includegraphics[width=\textwidth]{img/SocialNetworkABM_CONTROL.pdf}
	\captionof{figure}{Grafico cumulativo del modello con intervento non farmaceutico del controllore}
	\label{fig:abm_intervent}
\end{minipage}

Queste metriche sono metriche reali ma che nel modello vengono viste come un insieme unico, associabile 
ad una somma di medie dei valori di ogni metrica. Questa scelta porta a non avere un indice molto chiaro
ma permette comunque di avere un idea generale estremamente immediata dell'effetto che queste hanno sul modello.

Si nota come con l'applicazione di un insieme di contromisure, piu' o meno stringenti a seconda del periodo osservato, 
possiamo osservare come le curve epidemiologiche tendono ad appiattirsi, soprattutto quelle associate ai compartimenti 
\textbf{E, I}, le quali influenzano le curve \textbf{S, R}. In questo caso la ciclicita' del modello viene meno. 

Altro dato interessante e' il numero di VOC che e' diminuito rispetto al modello senza intervento.
Questo dipende generalmente dal comportamento delle contromisure. Infatti queste quando vengono applicate, 
oltre a influenzare il parametro \textbf{happiness}, vanno ad influenzare anche la matrice di flusso \ref{fig:migration matrix}
andando a ridurne i valori presenti. Questo fa si che oltre a dilazionare la diffusione della pandemia, 
va a dilazionare il comportamento della funzione che si occupa di generare una VOC \ref{fig:voc}. 

Questa infatti puo' attivarse sse nel nodo sono presenti individui infetti, in quanto non e' stata modellata la 
possibilita' che spontaneamente una nuova variante arrivi in un nuovo nodo senza un veicolo umano. Percio' 
applicare delle contromisure permette anche a contenere la diffusione di VOC nella popolazione osservata.

Si puo' osservare come la variabile di controllo \textbf{happiness} segua molto attentamente l'andamento sia delle 
contromisure che della pandemia, arrivando all'inizio della pandemia quando le contromisure sono piu' stringenti 
a diventare pressoche nulla. Successivamente, pur mantenendo un livello di contromisure sostenuto, la media
del valore di controllo tende ad alzarsi, seguendo il valore del compartimento \textbf{R}. Questo approccio sembra 
imitare il generale andamento della risposta che la popolazione italiana ha avuto alle prime misure di contenimento 
applicate realmente durante la pandemia da COVID-19.

\subsection{Intervento farmaceutico}
Il seguente grafico \ref{fig:abm_vaccine} mostra l'andamento delle curve del modello
quando questo viene eseguito tramite l'applicazione di una qualche tipologia di intervento farmaceutico.
Questo grafico dipende enormemente da quando viene trovato e successivamente applicato un intervento 
farmaceutico alla popolazione. Prima viene applicato un intervento farmaceutico piu' le curve si modificano 
e sono differenti dalll'andamento senza intervento \ref{fig:abm_no_intervent}, mentre piu' tempo passa piu'
il comportamento tende ad assomigliarsi.

\begin{minipage}{\linewidth}
	\centering
	\includegraphics[width=\textwidth]{img/SocialNetworkABM_VACCINE.pdf}
	\captionof{figure}{Grafico cumulativo del modello con intervento del controllore tramite interventi farmaceutici come ad esempio un vaccino}
	\label{fig:abm_vaccine}
\end{minipage}

Il grafico mostra come se disponibili e applicate repentinamente, l'utilizzo di contromisure farmaceutiche permette 
permette di ridurre repentinamente le curve senza andare ad intaccare la curva di happiness in maniera troppo sensibile.

\subsection{Intervento farmaceutico e non farmaceutico}
Il seguente grafico \ref{fig:abm_all} mostra l'andamento delle curve del modello
quando questo viene eseguito tramite l'applicazione sia di un intervento di tipo farmaceutico 
come ad esempio l'utillizo di un vaccino, che di contromisure non farmaceutiche.

\begin{minipage}{\linewidth}
	\centering
	\includegraphics[width=\textwidth]{img/SocialNetworkABM_ALL.pdf}
	\captionof{figure}{Grafico cumulativo del modello con intervento del controllore tramite vaccino e metodi di prevenzione non farmaceutici}
	\label{fig:abm_all}
\end{minipage}

In questo caso viene mostrato l'andamento delle curve tenendo in considerazione l'utilizzo di ogni mezzo
per prevenire e contrastare l'epidemia. L'uitilizzo combinato di mezzi farmaceutici e non permettono 
si appiattire notevolmente la curva di infetti andando a creare velocemente un immunita' di gruppo 
che rende la popolazione meno suscettibile alle varianti. Questo inoltre fa si che la \emph{happiness} media
del modello sia generalmente piu' alta anche quando vengono applicate delle contromisure che vanno ad 
intaccare la felicita' della popolazione (ad esempio un lockdown). 

Queste contromisure non farmaceutiche sono in genere meno stringenti e meno prolungate, permettendo quindi
alla popolazione di non avere un calo drastico della happiness generale come in figura \ref{fig:abm_intervent}.
Tuttavia dipendono fortemente da quando le contromisure farmaceutiche (il vaccino) vengono applicate
e con che efficacia questo riesce ad entrare in circolazione. 

Tuttavia questo dimostra come l'utilizzo di un vaccino a priori sia un metodo molto efficace per contrastare
una epidemia, e che ovviamente l'efficacia di questo dipenda molto e soprattutto da quando viene applicato 
alla popolazione. Successivamente si puo' notare come le misure non farmaceutiche di prevenzione e contrasto
dell'epidemia sono un mezzo efficace per controllare la diffusione dell'infezione, ma queste hanno un costo 
in termini sia di generale qualita' della vita, che anche economico, non indifferente come mostrato dai grafici 
precedenti e dall'esperienza diretta che si e' avuto con la pandemia da COVID-19. 

\subsection{Analisi di sensitività}
\begin{minipage}{\linewidth}
	\centering
	\includegraphics[width=\textwidth]{img/sa.pdf}
	\captionof{figure}{Grafico rappresentante l'analisi di sensitivita' del modello}
	\label{fig:sens_anal}
\end{minipage}
\section{Conclusioni}

In conclusione, i risultati ottenuti dimostrano  
l'efficacia delle misure di controllo, sia farmaceutiche che non 
farmaceutiche, nel contenere e ridurre la diffusione del virus. 
L'intervento farmaceutico, in particolare, si è dimostrato altamente 
efficace nel contrastare l'epidemia e nel prevenire il ciclo 
insostenibile di infezioni. Tuttavia, da queste simulazioni, è importante sottolineare che 
l'intervento farmaceutico da solo potrebbe non essere sufficiente per 
l'eliminazione completa dell'epidemia. Al contrario, l'adozione di una 
combinazione di interventi farmaceutici e non farmaceutici offre una 
protezione ottimale e una strategia più completa per la gestione delle 
pandemie.

Va notato che i tempi di calcolo delle simulazioni sono notevolmente 
influenzati dall' applicazione delle misure di controllo non farmaceutiche. 
Tuttavia, questa è un compromesso necessario per ottenere risultati 
accurati e realistici che tengano conto della complessità delle dinamiche 
sociali ed epidemiologiche. Inoltre, l'osservazione di oscillazioni 
nelle curve di felicità, causate dall'uso di misure di controllo cicliche, 
non rappresenta un problema insormontabile e può essere gestita 
adeguatamente.

In generale, il modello di simulazione dell'epidemia su una rete sociale,
unitamente all'utilizzo di modelli matematici compartimentali di tipo SEIR
con l'applicazione di un controllore gestito tramite Neural ODE, 
si configura come una piattaforma estremamente utile per lo studio degli 
effetti delle misure di controllo e per lo sviluppo di strategie ottimali 
per la gestione delle epidemie. I risultati ottenuti contribuiscono alla 
comprensione e alla preparazione per affrontare situazioni 
epidemiologiche complesse e dinamiche come quelle delle pandemie.

\subsection{Sviluppi Futuri}

\subsubsection{Perfezionamento del Controllore}

Un possibile sviluppo futuro di rilevanza è il perfezionamento del 
controllore, mirando a migliorarne sia le capacità di individuazione 
che l'applicazione delle policy. Inoltre, si potrebbe cercare di 
rendere le policy generate dal controllore più comprensibili per 
gli operatori umani, in modo da agevolare la loro traduzione e 
comprensione. Questa evoluzione apre la porta a un'approfondita 
analisi di causalità, che è un elemento di estrema importanza, 
come precedentemente discusso. In particolare, sarà necessario 
esplorare la relazione tra l'applicazione di specifiche 
contromisure e il loro impatto sulla riduzione di una pandemia.

Tale approccio si presenta come una sfida complessa ma altamente 
promettente, poiché potrebbe rappresentare un notevole 
avanzamento tecnologico.

\subsubsection{Miglioramento della Funzione di Happiness}

Un'altra importante area di sviluppo futuro riguarda la funzione 
di happiness, che gioca un ruolo cruciale nel controllo del 
controllore. Attualmente, questa funzione è utilizzata in 
modo semplice per evitare l'attivazione di contromisure insostenibili 
nella vita reale. Tuttavia, è possibile migliorarla ulteriormente, 
tenendo conto dell'effettivo impatto che ciascuna tipologia di 
contromisura può avere sulla vita delle persone, sia nel breve 
che nel lungo periodo. Questo miglioramento potrebbe rappresentare 
una pietra miliare per l'accuratezza delle policy sviluppate, 
fornendo un quadro di valutazione più preciso.

Tuttavia, anche questa area di sviluppo richiede un'analisi 
attenta della causalità degli eventi, in particolare per stimare 
l'effetto delle contromisure sull'umore complessivo di una popolazione. 
L'umore può variare a causa di numerosi fattori, comprese le 
contromisure e gli stimoli esterni, e l'analisi richiederà il 
supporto di diversi benchmark con modelli di simulazione diversificati.

\subsubsection{Perfezionamento Generale del Modello}

Il modello attualmente implementato si basa su assunzioni 
relativamente realistiche, derivanti da osservazioni empiriche. 
Tuttavia, il modello potrebbe beneficiare di un maggiore grado 
di generalizzazione e flessibilità per adattarsi a una varietà di 
problemi. Ciò comporterebbe la necessità di superare alcune rigidità 
strutturali attuali che ne limitano l'utilità in contesti diversi.

\subsubsection{Ottimizzazione Generale}

Un ulteriore sviluppo importante riguarda l'ottimizzazione 
generale del codice. Attualmente, il codice potrebbe essere 
ulteriormente migliorato in termini di efficienza, scalabilità e 
ottimizzazione delle performance. Questo sforzo mirerebbe a rendere 
il codice più performante e adattabile alle esigenze future.

% discussione delle conclusioni e possibili ringraziamenti
% \include{./txt/conclusions/conclusion.tex}
% \include{./txt/conclusions/ringraziamenti}

\nocite{*}

% \bibliographystyle{IEEEtran}
% \bibliographystyle{alpha}
\bibliographystyle{plain}
\bibliography{reference.bib}

\appendix
\section{Approcci scartati}

\subsection{Modello ad Agente su spazio continuo}
L'idea alla base era qualla di modellare una popolazione tramite l'utilizzo di 
uno spazio del modello di tipo continuo. Gli agenti sarebbero poi stati modellati come 
individui singoli e reali, in quanto entita' effettivamente attive all'interno 
dello spazio. Ci si puo' immaginare lo spazio del modello come una grande griglia 
definita da coordinate \emph{continue} di una certa dimensione \emph{$N \times M$}. All'interno 
di questa griglia vengono posizionati randomicamente gli agenti, i quali vengono 
rappresentati graficamente come dei punti di differenti colori.

\begin{minipage}{\linewidth}
    \centering
    \includegraphics[width=\textwidth]{img/ball-covid.png}
    \captionof{figure}{Esempio del modello modellato su spazio continuo}
    \label{fig:ball_covid}
\end{minipage}

Questo approccio utilizzava una metodologia similare a quella delle palline da biliardo 
per modellare l'interazio tra agenti all'interno dello spazio del modello. Ogni agente poteva
infettare in maniera casuale un suo vicino sse i due avevano un interazione. Ogni interazione
era modellata similmente ad un urto elastico \cite{wiki:Urto_elastico} tra corpi. 
Questo non era tuttavia un simulatore di urti elastici affidabili, bensi' un approccio che prendeva
spunto da esso, in particolare dal gioco del biliardo. 

In questo modo il comportamento complessivo del modello poteva permettere l'osservarsi 
di un andamento simile a quanto ci si aspetterebbe da un sistema simile.

\begin{minipage}{\linewidth}
    \centering
    \includegraphics[width=\textwidth]{img/plot_abm_continuousspace.pdf}
    \captionof{figure}{Esempio del comportamento delle curve nel modello continuo}
    \label{fig:seir_curve_continuous}
\end{minipage}

Sono state poi implementate svariate proprieta' come la proprieta' di essere individuati 
dopo un periodo di latenza come individui infetti e quindi essere confinati in quarantena, 
la quale era definita come una diminuzione nella probabilita' di infettare ed essere infettati 
perdendo la capacita' di muoversi.

Questa tipologia di approccio, quella Agent oriented, richiedeva una quantita' di risorse computazionali 
e di tempo estremamente elevato in relazione al numero di agenti presenti nel modello. Questo approccio 
inoltre e' stato valutato come troppo granulare e inadatto allo scopo del progetto. 
Si e' deciso quindi di adottare un approccio meno granulare e piu' flessibile 
andando ad un livello di astrazione piu' alto.

\subsection{Modello ad Agente con spazio a grafo e modellazione singolo agente}
L'idea e' nata per cercare di avere un controllo piu' granulare sullo spazio del modello e sulla sua 
evoluzione locale, non tanto degli agenti. Sono nati molteplici problemi, primo tra tutti quello del tempo impiegato
e successivamente quello del comportamento delle curve epidemiologiche che non rispettavano 
assolutamente il comportamento descritto dal modello deterministico SEIR, ma non in un modo 
ragionevole, bensi' in un modo completamente alieno. 

\begin{minipage}{\linewidth}
    \centering
    \includegraphics[width=\textwidth]{img/COMPARISON DIFFERENT R₀ VALUE_2023-06-29.pdf}
    \captionof{figure}{Comportamento modello ABM su spazio a grafo al variare del parametro $R_0$}
    \label{fig:strange_behaviour_R0_abm}
\end{minipage}

\begin{minipage}{\linewidth}
    \centering
    \includegraphics[width=\textwidth]{img/COMPARISON DIFFERENT R₀ VALUE_2023-06-29 (1).pdf}
    \captionof{figure}{Comportamento modello SEIR al variare del parametro $R_0$}
    \label{fig:strange_behaviour_R0_ode}
\end{minipage}

Si puo' facilmente osservare come il comportamento delle curve prenda un comportamento 
anomalo fin dalle prime variazioni del parametro $R_0$ per culminare con risultati 
completamente alieni. Il motivo alla base rimane tutt'ora ignoto e sconosciuto, tuttavia
esecuzioni differenti hanno mostrato dei comportamenti differenti, seppur altrettanto 
alieni. 

Questo comportamento pero' puo' essere descritto a grandi linee dalla seguente formula

\begin{minipage}{\linewidth}
    \centering
    \includegraphics[width=\textwidth]{img/rapporto_strano.png}
    \captionof{figure}{Formula che si occupa di descrivere il rapporto tra il comportamento del modello scartato e del modello SEIR. In particolare questa formula descrive il rapporto tra gli $R_0$}
    \label{fig:strange_behaviour_R0}
\end{minipage}

Bisogna tuttavia precisare come run successive abbiano portato a risultati differenti dei plot 
relativi al comportamento del modello ad agente. Il motivo rimane tutt'ora ignoto ma e' possibile
che uno degli attori in gioco possa essere la funzione \textbf{random} associata al campionamento 
degli individui che va a fare da mimica per il numero di contatti che avvengono ad ogni passo 
per ogni individuo infetto. Questo campionamento si basa principalmente sul calcolo di una 
distribuzione di \textbf{Poisson} \cite{wiki:Distribuzione_di_Poisson} con parametro $\lambda = R_0$.

In particolare la formula e' stata ricavata dopo aver fatto un confronto tra molteplici 
valori di $R_0$ e i risultati ottenuti sia dal modello SEIR che dal modello ABM. 
Successivamente e' stato calcolato \textbf{MSE} (Medium Square Error) per ogni coppia possibile
di risultati per cercare quelli tra loro piu' simili, andando ad ottenere quindi un insieme
di coppie. Queste poi sono state inserite in un metodo che calcolava una \textbf{polynomial fit} \cite{wiki:Polynomial_regression}
tra tutte le coppie di valori per vedere quale poteva essere la formula che governava la differenza 
di risultati. Da tenere a mente che in un caso normale questa differenza non dovrebbe esistere, 
o quanto meno se esiste dovrebbe essere trascurabile.

\begin{figure}[!hb]
	\centering
	\begin{subfigure}[b]{0.45\textwidth}
		\centering
		\includegraphics[width=\textwidth]{img/r0_range_test.png}
    	\caption{figure}{Range di valori di $R_0$ testati}
    	\label{fig:r0_range_test}
	\end{subfigure}
	\hfill
	\begin{subfigure}[b]{0.45\textwidth}
		\centering
		\includegraphics[width=\textwidth]{img/mse_r0.png}
		\caption{Calcolo MSE risultati SEIR - ABM}
		\label{fig:mse_r0}
	\end{subfigure}
\end{figure}

Da qui si ottiene la formula in figura \ref{fig:strange_behaviour_R0} la quale descrive approssimativamente, 
in base al grado del polinomio che si vuole andare a creare, la relazione che esiste tra le coppie 
di valori analizzate.

La motivazione per cui il modello ad agente si comporta in maniera cosi' inaspettata
rispetto a come dovrebbe non e' stato chiaro, e in letteratura non sembra esserci alcun 
articolo che ne parli in maniera approfondita, e per questo ho deciso di abbandonare l'approccio.
Punto a favore e' stato anche il fatto che l'approccio con ABM era estremaente esoso in termini 
di risorse computazionali e tempistiche. Questo problema e' insito in ogni tipo di simulazione, 
tuttavia affiancato allo stravagante problema comparso e descritto sopra, e' stato 
decisivo per il cambio drastico di approccio. 

\subsection{Controllore Ipopt}
Ipopt (Interior Point OPTimizer) \cite{Wächter2006} e' un pacchetto software per 
l'ottimizzazione non lineare su larga scala. Questo pacchetto software e' realizzato 
per trovare delle soluzioni (locali) a problemi di ottimizzazione matematica nella forma:
$\min_{x \in R^n} f(x)$ tale che $g_L \leq g(x) \leq g_U$ e $x_L \leq x \leq x_U$, dove 
$f(x): R^n \rightarrow R$ e' la funzione obiettivo e $g(x): R^n \rightarrow R^m$ 
sono le funzioni di vincolo.

I vettori $g_L$ e $g_U$ denotano i limiti inferiore e superiore sui vincoli e i vettori
$x_L$ e $x_U$ sono i limiti delle variabili $x$. Le funzioni $f(x)$ e $g(x)$ possono essere 
sia non lineari che non convesse, ma la loro derivata seconda deve esistere e deve essere 
continua.

\begin{minipage}{\linewidth}
    \centering
    \includegraphics[width=\textwidth]{img/Comparison-of-Ipopt-performance-over-various-linear-solvers-using-the-two-dimensional.png}
    \captionof{figure}{Comparison of Ipopt performance over various linear solvers using the two-dimensional partial differential equation test problem set. \cite{unknown}}
    \label{fig:Ipopt_solver}
\end{minipage}

Ipopt e' scritto in \textbf{C++} ed e' stato rilasciato come software open source sotto la licenza \textbf{Eclipse Public License (EPL)}.

L'utilizzo della suite \textbf{Ipopt} e' stato fatto per l'applicazione di un sistema
di monitoraggio e intervento all'interno del modello di simulazione. Successivamente l'idea
di utilizzare un integrazione con la suite \textbf{SciML.ai} verra' sfruttata per 
aggiungere algoritmi di Machine Learning per rendere piu' realistico e consistente il modello 
nelle sue predizioni e scelte.

\begin{minipage}{\linewidth}
	\centering
	\includegraphics[width=\textwidth]{img/controller_ipopt.png}
	\captionof{figure}{Definizione del controllore tramite Ipopt}
	\label{fig:controller_ipopt}
\end{minipage}

L'approccio generale e' stato semplice ma efficace, in quanto viene definito un modello 
per raccogliere tutte le informazioni relative e necessarie per l'algoritmo di ottimizzazione
e successivamente vengono definite le regole che governano il comportamento del modello. 

Le regole in questione sono principalmente le stesse che sono usate per descrivere il modello SEIR
che viene utilizzato all'interno del modello ad agente.

\begin{minipage}{\linewidth}
	\centering
	\includegraphics[width=\textwidth]{img/controller_rules.png}
	\captionof{figure}{Definizione regole del modello del controller}
	\label{fig:controller_rules}
\end{minipage}

Essendo che le regole mostrate in figura \ref{fig:controller_rules} sono relative agli stati 
SEIR e l'idea alla base del controllore e' quella di ridurre quanto piu' possibile il numero di
infetti cumulati che ci sono all'interno del sistema, e' stato aggiunto uno stato che descrive
appunto questo stato aggiuntivo.

Successivamente vi sono delle regole su quanto il modello puo' impiegare in termini di risorse, 
le quali sono le nostre contromisure con relativo costo, dato dall'integrale del valore della nostra
contromisura applicata nel tempo.

\begin{minipage}{\linewidth}
	\centering
	\includegraphics[width=\textwidth]{img/controller_rules_1.png}
	\captionof{figure}{Definizione regole del modello del controller per le contromisure}
	\label{fig:controller_rules_1}
\end{minipage}

Infine viene ottimizzato il modello e ritornato il valore medio delle contromisure applicate
quando applicate.

\end{document}