\section{Introduzione}

L’utilizzo di metodi e tecniche sempre più sofisticati 
da parte della comunità scientifica, in particolar modo 
da parte di epidemiologi e medici, è sempre stato argomento 
di grande dibattito e interesse. Negli ultimi anni il mondo 
si è espanso esponenzialmente divenendo sempre più connesso,
incrementando drasticamente la probabiità che un virus 
affligga a livello mondiale la popolazione creando un 
disastro senza precedenti. 

La storia dell’uomo è costellata di epidemie, e solamente 
alcune tra loro si sono guadagnate il privilegio di essere 
ricordate e tra queste solamente una piccola parte ha ottenuto 
il primato di essere ricordata come una catastrofe. Forse 
è proprio questo che le ha rese così salde nell'immaginario 
comune aumentando la loro già imponente aurea di terrore.

Per fare degli esempi possiamo citare: la peste nera
\cite{wiki:Peste_nera} che a partire dalla 
metà del 14esimo secolo ritornò in Europa
uccidendo venti milioni di persone in soli sei anni, 
l'epidemia di tifo \cite{wiki:Tifo_esantematico} 
che non solo fu fatale durante il periodo delle crociate, 
ma anche durante la seconda guerra mondiale all'interno 
dei campi di concentramento nazisti, oppure le varie 
influenze, prima tra tutte quella spagnola \cite{wiki:Influenza_spagnola}
la quale nel periodo del primo dopo guerra, ovvero tra il 
1918 e il 1920, uccise 50 milioni di persone in tutto il mondo, 
oppure l'epidemia di AIDS \cite{wiki:Storia_della_pandemia_di_AIDS}, 
che ha all'attivo dal 1981 oltre 75 milioni di casi e 35 milioni di morti.

Se ci si sofferma un attimo pensando proprio a questo tipo 
di pandemia, quella influenzale, ci viene da tirare un 
sospiro di sollievo, in quanto oramai come cittadini del 
primo mondo l'idea di influenza non ci fa più così paura.
Eppure dovrebbe, e sfortunatamente lo abbiamo ricordato 
nel peggior modo possibile. 

La pandemia di SARS-CoV-2 \cite{wiki:Pandemia_di_COVID-19} 
scoppiata negli ultimi mesi del 2019 condizionò 
l’intera umanità per circa 3 anni, e nel momento che sto 
scrivendo queste righe continua a condizionarla. 
Questa pandemia si è macchiata di aver mietuto, 
allo stato attuale delle cose, quasi 7 milioni di vite accertate. 
Questa tragedia rimarrà impressa nella memoria umana 
in quanto capace di aver messo in crisi l'intero sistema
governativo mondiale, creando uno stato di allarme, panico e 
alle volte perfino isteria, che pochi altri avvenimenti 
sono stati in grado di fare. 

Osservando le statistiche proprie di questa epidemia, 
ciò che balza subito all’occhio è sicuramente il numero 
associato alle morti e agli infetti: quasi 7 milioni di 
morti e più di 700 milioni di infetti \cite{world_health_organization}. 
Numeri che da soli basterebbero a mettere a disagio 
qualsiasi lettore, eppure altri dati, nascosti a prima 
vista, possono fornire ulteriori macabre informazioni. 
Un esempio tra tutti è l’effetto che un epidemia del 
genere ha avuto sull’economia \cite{world_bank_group_2023}, 
portando disagi generalizzati ovunque. 

Legato al disagio economico vi è un altro dato preoccupante 
definito come poverty trap \cite{Bonds2009-sg} \cite{wiki:Cycle_of_poverty}. 
Questo fenomeno nasce in quegli ambienti in cui le 
condizioni di povertà economica e la prevalenza di malattie 
possono imprigionare una società in uno stato persistente 
di bassa sanità e sempre maggiore povertà; 
questo fenomeno ciclico si auto sostiene e prende il nome 
di positive feedback \cite{wiki:Positive_feedback}. 
Per ultimo, ma non per questo meno importante, l’effetto di 
una pandemia può portare ad instabilità governative le 
quali possono portare a un arretramento del sistema 
sanitario e di welfare \cite{https://doi.org/10.1002/epa2.1152}, 
ricadendo come sopra descritto all’interno del fenomeno di positive feedback.

Questi sono solamente alcuni esempi dei problemi che 
possono sorgere, e che sono sorti con lo scoppio di una 
pandemia globale come è stata quella del COVID-19. 
Per questo la comunità scientifica, in particolare gli 
epidemiologi cercano soluzioni sempre più efficaci e 
accurate per prevenire, arginare e contrastare avvenimenti 
del genere.

L’epidemiologia è una disciplina nata di recente 
evolutasi insieme alle esigenze della società ogni 
qualvolta una nuova emergenza sanitaria faceva irruzione 
nella quotidianità. La prima definizione di epidemiologia 
è stata data da Lilienfeld \cite{10.1371/journal.pone.0208442} 
nel 1978, e cita: 

\begin{quotation}
    \emph{l’epidemiologia è un modo di ragionare riguardo le 
    malattie, e si occupa di effettuare inferenza biologica 
    derivata dall’osservazione di fenomeni patologici 
    all’interno di una popolazione.}
\end{quotation}

Con il tempo questa definizione ha subito molti cambiamenti, 
derivati anche e soprattutto dall’espansione degli ambiti 
relazionati all’epidemiologia; con l’aggiunta ad esempio 
della farmacoepidemiologia, dell’epidemiologia molecolare 
e dell’epidemiologia genetica. Non solo, ambiti come etica, 
filosofia ed epistemiologia sono estremamente importanti 
ed influenti nella crescita e sviluppo di questa materia \cite{10.1371/journal.pone.0208442}. 

Attualmente con il termine epidemiologia si intende la 
disciplina biomedica che studia la distribuzione e la 
frequenza delle malattie ed eventi di rilevanza sanitaria 
nella popolazione. L’epidemiologia si occupa di analizzare 
le cause, il decorso e le conseguenze delle malattie \cite{wiki:Epidemiologia}. 
Secondo Last et al (1998) l’epidemiologia è definita come: 

\begin{quotation}
    \emph{lo studio della distribuzione e dei determinanti 
    delle situazioni o degli eventi collegati alla salute 
    in una specifica popolazione, e l'applicazione di 
    questo studio al controllo dei problemi di salute.}
\end{quotation}

Uno degli strumenti più utilizzati in epidemiologia sono le 
simulazioni software \cite{wiki:Simulation_software}. 
Dato lo scopo dell’epidemiologia, questa necessita di 
avvalersi di modelli matematici \cite{doi:10.4161/viru.24041} 
che aiutano i ricercatori a trarre conclusioni sul sistema 
che analizza. Sistemi simili vengono definiti come complessi,
\cite{Galea2009-lj} \cite{Ladyman2013} ovvero sistemi 
dinamici a multicomponenti che tipicamente interagiscono 
tra loro, e che sono descrivibili tramite modelli matematici. 

I primi modelli chiamati compartimentali \cite{Bjornstad2020} 
basano il proprio funzionamento sullo studio di gruppi di 
individui disgiunti che interagiscono tra loro, 
analizzabili tramite un sistema di equazioni ordinarie 
differenziali (ODE) \cite{Brauer2008}. 
Questo approccio è stato teorizzato da Kermack e McKendrick 
nel 1927 applicando una modellazione matematica al comportamento 
delle malattie infettive su un gruppo di individui, tenendo in 
considerazione la variabile del tempo. Da qui è nato il famoso modello 
Susceptible-Infectious-Recovered (SIR) \cite{wiki:Compartmental_models_in_epidemiology}, 
il quale tuttora viene utilizzato ampiamente insieme alle sue varianti. 
Successivamente con l’unione di svariate discipline come: 
la teoria dei giochi, i sistemi complessi, i comportamento emergente, 
la sociologia computazionale, i sistemi multiagente e 
la programmazione evoluzionaria sono nati i modelli ad agente 
\cite{7822080} \cite{Bissett2021}, modelli computazionali 
autonomi per la simulazione di sistemi complessi. 

Grazie alla loro teorizzazione negli anni 1940 ma soprattutto 
grazie al loro uso concreto dagli anni 1990 il mondo della 
simulazione ha avuto un grande balzo in avanti. 
Ciò che rende questi sistemi molto flessibili e potenti è la 
capacità di far emergere spontaneamente dei comportamenti 
complessi da un insieme di regole semplici. 
Ovviamente però la richiesta di risorse e capacità 
computazionale è estremamente elevata rispetto alla 
controparte di modelli puramente matematici. 

Ogni approccio ha dei pro e dei contro e la vera potenza di 
questi modelli sta principalmente nella acutezza di chi deve 
poi farne uso. In base ai compromessi e le assunzioni fatte 
durante la fase di modellazione ogni approccio può rivelarsi 
vincente. Uno dei compromessi maggiori che viene generalmente 
applicato a questi modelli è quello della discretizzazione 
dell’ambiente \cite{KONSTANTINOUDIS2020100319}. 
La realtà è a tutti gli effetti un insieme continuo di eventi, 
ma essendo tutti i dispositivi di calcolo discreti è 
impossibile simulare avvenimenti continui 
(siano essi nel tempo o nello spazio) in maniera diretta e 
perciò bisogna fare dei compromessi, trasformando il proprio 
spazio di lavoro in uno più adatto alle macchine che lo devono 
simulare. 

Un altra assunzione generalista e approssimativa per 
definizione, ma necessaria per la costruzione di un modello 
che analizzi il decorso di un’epidemia all’interno di una 
società, è quella legata alla tipologia di comportamento che 
verrà mostrato dagli esseri umani in condizioni di pericolo 
\cite{Tracy2018-lc} \cite{El-Sayed2012-ac}. 
Queste sono solo alcune delle assunzioni e compromessi che 
bisogna fare quando ci si approccia al mondo della simulazione. 
Tuttavia applicare delle assunzioni, alle volte anche forti e 
controintuitive, non sempre è sinonimo di errore. 
Alle volte tramite lo studio della causalità degli eventi 
\cite{Galea2009-lj} \cite{Parascandola2001-kw} è possibile 
astrarre un set minimale di assunzioni che se applicate danno 
la capacità al modello di rappresentare molto bene il 
comportamento desiderato. 

Certamente ci saranno alcune discrepanze soprattutto in casi 
estremi, ma è un’eventualità che viene tenuta in considerazione 
ogni qualvolta si parla di simulazione,e che non è possibile 
eliminare del tutto. 

\begin{quotation}
    \emph{La causalità o causa effetto, è quell’influenza per cui un evento, 
    un processo, uno stato o un oggetto contribuiscono nella produzione 
    di un nuovo evento, processo, stato o effetto, dove la causa è 
    parzialmente responsabile dell’effetto e l’effetto è parzialmente 
    dipendente dalla causa.}
\end{quotation}

A prima vista non sembra una tematica molto complessa o di 
difficile approccio, complice il fatto che in quanto esseri 
umani siamo una specie che si è evoluta per trovare una 
correlazione tra gli eventi, ma correlazione non significa, 
e soprattutto non implica, causalità \cite{Altman2015}. 
Questo tema si riferisce all’incapacità legittima di dedurre 
la relazione di causa - effetto tra due eventi o variabili 
solamente sulla base di una osservazione della loro 
associatività o correlazione. 

Diventa chiaro come una delle parti più complesse 
dell’epidemiologia sia proprio quella di stabilire le cause di 
un dato fenomeno e comprendere come un determinato intervento 
su di esse influenzi quest’ultimo \cite{Galea2009-lj} 
\cite{Parascandola2001-kw}. 

Questa rapida introduzione all’epidemiologia e alla simulazione 
di sistemi complessi tramite l’utilizzo in particolare di 
modelli ad agente sarà l’argomento cardine dell’intero 
elaborato. Nelle sezioni successive verrà proposta un analisi 
dello stato dell’arte dell’attuale 
panorama epidemiologico e simulativo, con alcune rapide 
digressioni sul problema della discretizzazione, problema 
assai sentito nell’ambito della simulazione. 

Successivamente verranno portati alla luce alcuni esempi e 
modelli di sistemi ad agente che si occupano di modellare 
differenti tipi di epidemie, tutte caratterizzate però dal 
fatto di essere epidemie infettive a diffusione principalmente 
aerea, come ad esempio l’ebola, l’aviaria e l’influenza. 

Successivamente vi sarà un analisi più dettagliata riguardo la 
pandemia da COVID-19 che recentemente ci ha colpito. 
Infine verranno analizzati alcuni metodi di monitoraggio di 
queste simulazioni con un focus su alcuni metodi di intervento.