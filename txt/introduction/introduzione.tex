\section{Introduzione}
La storia umana è segnata da epidemie, ma solo alcune di esse hanno 
lasciato un'impronta duratura nella memoria collettiva a causa delle loro 
conseguenze catastrofiche. Tra queste, alcune delle più note includono la 
peste nera che nel XIV secolo mieté venti milioni di vite in Europa in 
soli sei anni, l'epidemia di tifo durante le crociate e la Seconda 
Guerra Mondiale, l'influenza spagnola che causò 50 milioni di morti tra 
il 1918 e il 1920, e l'epidemia di AIDS, che ha colpito oltre 75 milioni 
di persone e causato 35 milioni di decessi dal 1981.

Oggi, l'influenza stagionale non suscita più lo stesso timore nei 
cittadini dei paesi sviluppati, ma la pandemia di COVID-19, iniziata alla 
fine del 2019, ha condizionato l'umanità per tre anni e continua a farlo, 
causando finora quasi 7 milioni di vittime accertate. Questa pandemia 
rimarrà impressa nella memoria umana poiché ha messo in crisi l'intero 
sistema governativo globale, causando allarmi, panico e, talvolta, 
isteria, come pochi altri eventi sono stati in grado di fare.

Esaminando le statistiche di questa epidemia, i numeri relativi ai decessi 
e agli infetti (quasi 7 milioni di morti e quasi 800 milioni di infetti) 
da soli sono sufficienti a preoccupare qualsiasi lettore. Tuttavia, ci 
sono altri dati meno evidenti che forniscono informazioni altrettanto 
inquietanti, come l'impatto economico globale e il concetto di 
``poverty trap'', un circolo vizioso in cui la povertà e le malattie 
perpetuano un ciclo di bassa salute e crescente povertà.

Questi sono solo alcuni dei problemi che possono emergere durante una 
pandemia globale come quella del COVID-19, e quindi la comunità 
scientifica, in particolare gli epidemiologi, cerca costantemente 
soluzioni più efficaci ed accurate per prevenire, contenere e gestire 
eventi di questo genere.

L'epidemiologia è una disciplina relativamente giovane che si è evoluta 
per affrontare emergenze sanitarie. La sua definizione originale risale 
al 1978 \cite{Frerichs1978-cn} e cita: 
\begin{quote}
    	``Epidemiology is the study of the prevalence and dynamics of 
		stages of health in populations.''
\end{quote}
ma nel corso del tempo è cresciuta e si è adattata alle esigenze 
della società. 

Attualmente \cite{ward2012oxford}, l'epidemiologia è definita come:
\begin{quote}
    ``Epidemiology is the study of the distribution and determinants of 
	health-related states or events (including disease), and the 
	application of this study to the control of diseases and other 
	health problems. Various methods can be used to carry out 
	epidemiological investigations: surveillance and descriptive studies 
	can be used to study distribution; analytical studies are used to 
	study determinants.''
\end{quote}

Uno strumento ampiamente utilizzato in epidemiologia è la simulazione 
tramite software, che si basa su modelli matematici per trarre conclusioni 
sui sistemi analizzati. Questi sistemi, spesso definiti complessi, 
coinvolgono molteplici componenti che interagiscono tra loro.

I primi modelli epidemiologici erano basati sul paradigma compartimentale, 
dove gruppi di individui separati interagivano tra loro e potevano essere 
descritti mediante equazioni differenziali ordinarie (ODE). Tra i primi 
modelli utilizzanti le ODEs troviamo il così detto 
``Susceptible-Infected-Recovered'' (SIR) sviluppato da 
Kermack e McKendrick nel 1927 \cite{kermack1927contribution}. 
Successivamente, sono emersi i modelli ad agente, che consentono la 
simulazione di sistemi complessi tramite l'interazione di entità autonome.

La simulazione ha fatto notevoli progressi dagli anni '90 grazie 
all'espansione delle risorse computazionali. Tuttavia, i modelli di 
simulazione richiedono compromessi e semplificazioni, ad esempio la 
discretizzazione degli ambienti di simulazione, ma permettono lo 
studio di fenomeni definiti \emph{emergenti}. Questi sono comportamenti 
dell'intero sistema che non possono venire predetti osservando 
il comportamento dei singoli agenti. Inoltre, è necessario considerare 
il comportamento umano in situazioni di pericolo, il che può 
essere complesso da modellare.

L'identificazione delle cause di un fenomeno e la comprensione di come un 
intervento influenzi tale fenomeno rappresentano una delle sfide più 
complesse dell'epidemiologia. La causalità, ossia la relazione di 
causa-effetto tra eventi o variabili, non è sempre evidente e richiede 
un'analisi rigorosa. La correlazione tra eventi non implica 
necessariamente causalità.

Le sezioni successive analizzeranno lo stato dell'arte dell'epidemiologia 
e della simulazione, con un focus sulla pandemia da COVID-19 e sui metodi 
di monitoraggio e intervento nelle simulazioni epidemiologiche.