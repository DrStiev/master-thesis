\section{Sviluppi Futuri}

\subsection{Perfezionamento del Controllore}

Un possibile sviluppo futuro di rilevanza è il perfezionamento del 
controllore, mirando a migliorarne sia le capacità di individuazione 
che l'applicazione delle policy. Inoltre, si potrebbe cercare di 
rendere le policy generate dal controllore più comprensibili per 
gli operatori umani, in modo da agevolare la loro traduzione e 
comprensione. Questa evoluzione apre la porta a un'approfondita 
analisi di causalità, che è un elemento di estrema importanza, 
come precedentemente discusso. In particolare, sarà necessario 
esplorare la relazione tra l'applicazione di specifiche 
contromisure e il loro impatto sulla riduzione di una pandemia.

Tale approccio si presenta come una sfida complessa ma altamente 
promettente, poiché potrebbe rappresentare un notevole 
avanzamento tecnologico.

\subsection{Miglioramento della Funzione di Happiness}

Un'altra importante area di sviluppo futuro riguarda la funzione 
di happiness, che gioca un ruolo cruciale nel controllo del 
controllore. Attualmente, questa funzione è utilizzata in 
modo semplice per evitare l'attivazione di contromisure insostenibili 
nella vita reale. Tuttavia, è possibile migliorarla ulteriormente, 
tenendo conto dell'effettivo impatto che ciascuna tipologia di 
contromisura può avere sulla vita delle persone, sia nel breve 
che nel lungo periodo. Questo miglioramento potrebbe rappresentare 
una pietra miliare per l'accuratezza delle policy sviluppate, 
fornendo un quadro di valutazione più preciso.

Tuttavia, anche questa area di sviluppo richiede un'analisi 
attenta della causalità degli eventi, in particolare per stimare 
l'effetto delle contromisure sull'umore complessivo di una popolazione. 
L'umore può variare a causa di numerosi fattori, comprese le 
contromisure e gli stimoli esterni, e l'analisi richiederà il 
supporto di diversi benchmark con modelli di simulazione diversificati.

\subsection{Perfezionamento Generale del Modello}

Il modello attualmente implementato si basa su assunzioni 
relativamente realistiche, derivanti da osservazioni empiriche. 
Tuttavia, il modello potrebbe beneficiare di un maggiore grado 
di generalizzazione e flessibilità per adattarsi a una varietà di 
problemi. Ciò comporterebbe la necessità di superare alcune rigidità 
strutturali attuali che ne limitano l'utilità in contesti diversi.

\subsection{Ottimizzazione Generale}

Un ulteriore sviluppo importante riguarda l'ottimizzazione 
generale del codice. Attualmente, il codice potrebbe essere 
ulteriormente migliorato in termini di efficienza, scalabilità e 
ottimizzazione delle performance. Questo sforzo mirerebbe a rendere 
il codice più performante e adattabile alle esigenze future.