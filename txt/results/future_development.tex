\section{Sviluppi Futuri}
\subsection{Perfezionamento Controllore}
Uno sviluppo futuro può essere il perfezionamento del controllore sia nelle sue capacità di individuazione
e applicazione delle policy, sia nella chiarezza di ritornare delle policy \emph{human-understandable} così 
da essere facilmente tradotte e comprese anche da un operatore umano. Questo sviluppo apre le porte allo studio 
di causalità, elemento molto importante trattato in apertura dell'elaborato in quanto bisognerà cercare di capire 
una relazione che vi è tra l'applicazione di una specifica contromisura e il suo apporto alla diminuzione di una 
pandemia. 

Questo approccio è veramente complesso ma di estremo interesse in quanto potrebbe risultare in un balzo 
tecnologico non indifferente. 

\subsection{Miglioramento funzione happiness}
Un'altro grande scoglio definito come assunzione è la funzione di happiness, responsabile del 
controllo della funzione Controllore. Questa funzione attualmente serve solamente come brutale asticella per evitare 
di cadere all'interno di un ciclo disfunzionale di contromisure insostenibili nella vita reale. Tuttavia 
se venisse migliorato tenendo conto di quanto veramente ogni topologia di contromisura possa impattare sulla vita 
delle persone, sia sul breve che sul lungo periodo, questo potrebbe essere una pietra miliare per l'accuratezza delle 
policy sviluppate in quanto si avrebbe un valore di riscontro molto più accurato. 

Tuttavia anche questa sezione richiede un attenta analisi della causalità degli eventi, in particolare stimare quanto in 
generale una contromisura possa impattare l'umore di una popolazione. Umore il quale varia in maniera sia per via delle contromisure 
che anche per stimoli esterni alle volte neanche immaginabili. Ad aiuto sarebbe comodo avere a corredo molteplici \emph{benchtest} caratterizzati 
da differenti modelli di simulazione; partendo da quello più granulare per comprendere il comportamento emergente relativo 
al livello generale di soddisfazione di una popolazione dato un ambiente dinamico e auto influenzante, fino ad arrivare a mettere 
tutto quanto ottenuto insieme per avere una simulazione macroscopica in grado di offrire risultati affidabili.

\subsection{Perfezionamento generale modello}
Il modello implementato si basa su una serie di assunzioni relativamente ragionevoli, date dall'osservazione empirica
di quanto si voleva andare a modellare. L'osservazione e i dati raccolti durante tutto il periodo COVID-19 sono stati 
utili per ottenere un modello che possa essere abbastanza veritiero. Ciò nonostante questo modello attualmente non si presta
molto bene alla generalizzazione, ponendo il fianco a delle rigidità strutturali che ne compromettono l'uso. 

è quindi buona pratica quella di avere un modello che ben si adatti e generalizzi ai vari problemi cui viene sottoposto.

\subsection{Ottimizzazione generale}
Il codice sviluppato sicuramente ha nampio margine di miglioramento soprattutto sotto l'aspetto delle performance generali. 
In futuro è un nobile obiettivo quello di renderlo più efficiente, scalabile e ottimizzato possibile.

