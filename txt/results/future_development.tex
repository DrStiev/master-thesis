\section{Conclusioni}

In conclusione, i risultati ottenuti dimostrano  
l'efficacia delle misure di controllo, sia farmaceutiche che non 
farmaceutiche, nel contenere e ridurre la diffusione del virus. 
L'intervento farmaceutico, in particolare, si è dimostrato altamente 
efficace nel contrastare l'epidemia e nel prevenire il ciclo 
insostenibile di infezioni. Tuttavia, da queste simulazioni, è importante sottolineare che 
l'intervento farmaceutico da solo potrebbe non essere sufficiente per 
l'eliminazione completa dell'epidemia. Al contrario, l'adozione di una 
combinazione di interventi farmaceutici e non farmaceutici offre una 
protezione ottimale e una strategia più completa per la gestione delle 
pandemie.

Va notato che i tempi di calcolo delle simulazioni sono notevolmente 
influenzati dall' applicazione delle misure di controllo non farmaceutiche. 
Tuttavia, questa è un compromesso necessario per ottenere risultati 
accurati e realistici che tengano conto della complessità delle dinamiche 
sociali ed epidemiologiche. Inoltre, l'osservazione di oscillazioni 
nelle curve di felicità, causate dall'uso di misure di controllo cicliche, 
non rappresenta un problema insormontabile e può essere gestita 
adeguatamente.

In generale, il modello di simulazione dell'epidemia su una rete sociale,
unitamente all'utilizzo di modelli matematici compartimentali di tipo SEIR
con l'applicazione di un controllore gestito tramite Neural ODE, 
si configura come una piattaforma estremamente utile per lo studio degli 
effetti delle misure di controllo e per lo sviluppo di strategie ottimali 
per la gestione delle epidemie. I risultati ottenuti contribuiscono alla 
comprensione e alla preparazione per affrontare situazioni 
epidemiologiche complesse e dinamiche come quelle delle pandemie.

\subsection{Sviluppi Futuri}

\subsubsection{Perfezionamento del Controllore}

Un possibile sviluppo futuro di rilevanza è il perfezionamento del 
controllore, mirando a migliorarne sia le capacità di individuazione 
che l'applicazione delle policy. Inoltre, si potrebbe cercare di 
rendere le policy generate dal controllore più comprensibili per 
gli operatori umani, in modo da agevolare la loro traduzione e 
comprensione. Questa evoluzione apre la porta a un'approfondita 
analisi di causalità, che è un elemento di estrema importanza, 
come precedentemente discusso. In particolare, sarà necessario 
esplorare la relazione tra l'applicazione di specifiche 
contromisure e il loro impatto sulla riduzione di una pandemia.

Tale approccio si presenta come una sfida complessa ma altamente 
promettente, poiché potrebbe rappresentare un notevole 
avanzamento tecnologico.

\subsubsection{Miglioramento della Funzione di Happiness}

Un'altra importante area di sviluppo futuro riguarda la funzione 
di happiness, che gioca un ruolo cruciale nel controllo del 
controllore. Attualmente, questa funzione è utilizzata in 
modo semplice per evitare l'attivazione di contromisure insostenibili 
nella vita reale. Tuttavia, è possibile migliorarla ulteriormente, 
tenendo conto dell'effettivo impatto che ciascuna tipologia di 
contromisura può avere sulla vita delle persone, sia nel breve 
che nel lungo periodo. Questo miglioramento potrebbe rappresentare 
una pietra miliare per l'accuratezza delle policy sviluppate, 
fornendo un quadro di valutazione più preciso.

Tuttavia, anche questa area di sviluppo richiede un'analisi 
attenta della causalità degli eventi, in particolare per stimare 
l'effetto delle contromisure sull'umore complessivo di una popolazione. 
L'umore può variare a causa di numerosi fattori, comprese le 
contromisure e gli stimoli esterni, e l'analisi richiederà il 
supporto di diversi benchmark con modelli di simulazione diversificati.

\subsubsection{Perfezionamento Generale del Modello}

Il modello attualmente implementato si basa su assunzioni 
relativamente realistiche, derivanti da osservazioni empiriche. 
Tuttavia, il modello potrebbe beneficiare di un maggiore grado 
di generalizzazione e flessibilità per adattarsi a una varietà di 
problemi. Ciò comporterebbe la necessità di superare alcune rigidità 
strutturali attuali che ne limitano l'utilità in contesti diversi.

\subsubsection{Ottimizzazione Generale}

Un ulteriore sviluppo importante riguarda l'ottimizzazione 
generale del codice. Attualmente, il codice potrebbe essere 
ulteriormente migliorato in termini di efficienza, scalabilità e 
ottimizzazione delle performance. Questo sforzo mirerebbe a rendere 
il codice più performante e adattabile alle esigenze future.